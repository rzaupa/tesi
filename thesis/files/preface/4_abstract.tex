\cleardoublepage
\phantomsection
\pdfbookmark{Compendio}{Compendio}
\begingroup
\let\clearpage\relax
\let\cleardoublepage\relax

\chapter*{Sommario}

Il presente documento descrive il lavoro svolto durante il periodo di stage del laureando \myName presso l'azienda \myCompany. Tale periodo, svolto alla conclusione del percorso di studi triennale in Informatica presso l'Università degli Studi di Padova, ha avuto una durata complessiva di \myHours ore. \\
Gli obiettivi principali del progetto hanno riguardato l'analisi e l'utilizzo dei servizi AWS per l'addestramento di modelli di Intelligenza Artificiale (AI), finalizzati alla classificazione e all'estrapolazione automatica delle informazioni contenute nelle mail PEC (Poste Elettroniche Certificate). Durante lo stage, è stata eseguita un'analisi dettagliata dei requisiti applicativi e tecnici necessari per implmentare una soluzione efficace e robusta. \\
L'attività di sviluppo ha incluso l'utilizzo di un modello di apprendimento automatico capace di analizzare il contenuto delle PEC importate, assegnando loro categorie appropriate basate su criteri come mittente, destinatario, data e argomento. In parallelo, è stato esplorato l'utilizzo di algoritmi avanzati di IA in grado di adattarsi e migliorare le prestazioni del modello attraverso l'apprendimento continuo dai dati e dai feedback ricevuti. \\
Infine, si è considerata l'integrazione con un sistema documentale per l'archiviazione automatica delle PEC, con la creazione dei metadati necessari e il loro posizionamento nella corretta categoria di appartenenza. Questi aspetti desiderabili, sebbene non obbligatori, hanno rappresentato un'opportunità di estendere la funzionalità del sistema, migliorando ulteriormente l'efficienza e l'accuratezza dell'archiviazione delle PEC. \\

\endgroup
\vfill
