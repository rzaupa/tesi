% Acronyms
\newacronym[description={\glslink{aig}{Artificial Intelligence}}]
{ai}{AI}{Artificial Intelligence}

\newacronym[description={\glslink{apig}{Application Programming Interface}}]
{api}{API}{Application Program Interface}

\newacronym[description={\glslink{awsg}{Amazon Web Services}}]
{aws}{AWS}{Amazon Web Services}

\newacronym[description={\glslink{ictg}{Information and Communication Technology}}]
{ict}{ICT}{Information and Communication Technology}

\newacronym[description={\glslink{idpg}{Intelligence Document Processing}}]
{idp}{IDP}{Intelligence Document Processing}

\newacronym[description={\glslink{mlg}{Machine Learning}}]
{ml}{ML}{Machine Learning}

\newacronym[description={\glslink{nlpg}{Natural Language Processing}}]
{nlp}{NLP}{Natural Language Processing}

\newacronym[description={\glslink{ocrg}{Optical Character Recognition}}]
{ocr}{OCR}{Optical Character Recognition}

\newacronym[description={\glslink{pecg}{Posta Elettronica Certificata}}]
{pec}{PEC}{Posta Elettronica Certificata}

\newacronym[description={\glslink{sdkg}{Software Development Kit}}]
{sdk}{SDK}{Software Development Kit}

\newacronym[description={\glslink{umlg}{Unified Modeling Language}}]
{uml}{UML}{Unified Modeling Language}

% Glossary entries
\newglossaryentry{agileg}{
    name=\glslink{agileg}{Agile},
    text = Agile,
    sort = agile,
    description ={Nell'ambito dell'ingegneria del software con il termine Agile si intende un insieme di metodi di sviluppo del software basati su processi iterativi e incrementali, dove i requisiti e le soluzioni si evolvono attraverso la collaborazione tra team auto-organizzati e interfunzionali}
}

\newglossaryentry{aig}{
    name=\glslink{ai}{AI},
    text = AI,
    sort = ai,
    description = {Per artificial Intelligence (AI) si intende l'insieme di tecnologie e metodi che permettono ai computer di eseguire attività che richiedono intelligenza umana, come il riconoscimento di immagini, il riconoscimento vocale, la traduzione automatica, ecc}
}
% da mettere :
% idp e scalabilità
% Glossary entries
\newglossaryentry{apig} {
    name=\glslink{api}{API},
    text=Application Program Interface,
    sort=api,
    description={In informatica con il termine \emph{API} si indica ogni insieme di procedure disponibili al programmatore, di solito raggruppate a formare un set di strumenti specifici per l'espletamento di un determinato compito all'interno di un certo programma. La finalità è ottenere un'astrazione, di solito tra l'hardware e il programmatore o tra software a basso e quello ad alto livello semplificando così il lavoro di programmazione}
}

\newglossaryentry{awsg} {
    name=\glslink{aws}{AWS},
    text=AWS,
    sort=aws,
    description={Amazon Web Services (AWS) è una piattaforma di servizi cloud che offre potenza di calcolo, storage di database, distribuzione di contenuti e altre funzionalità per aiutare le imprese a scalare e crescere}
}

\newglossaryentry{gitg}{
    name=\glslink{gitg}{Git},
    text=Git,
    sort=git,
    description={Git è un sistema di controllo di versione distribuito gratuito e open source progettato per gestire tutto, dai piccoli ai grandi progetti, con velocità ed efficienza}
}

\newglossaryentry{ictg} {
    name=\glslink{ict}{ICT},
    text=ICT,
    sort=ict,
    description={Con il termine Information and Communication Technology (ICT) si intende l'insieme delle tecnologie informatiche e telematiche utilizzate per la gestione delle informazioni e la comunicazione}
}

\newglossaryentry{idpg}{
    name=\glslink{idp}{IDP},
    text=IDP,
    sort=idp,
    description={Con il termine Intelligence document processing (IDP) si intende l'insieme di tecnologie che permettono di estrarre informazioni da documenti cartacei o digitali, elaborarle e trasformarle in dati strutturati}
}

\newglossaryentry{mlg}{
    name=\glslink{ml}{ML},
    text=ML,
    sort=ml,
    description={Per Machine Learning (ML) si intende un insieme di tecniche e algoritmi che permettono ai computer di apprendere dai dati e di migliorare le prestazioni in base all'esperienza, senza essere esplicitamente programmati. Il machine learning si basa su modelli statistici e matematici che permettono di fare previsioni o decisioni in base ai dati analizzati}
}

\newglossaryentry{nlpg}{
    name=\glslink{nlp}{NLP},
    text=NLP,
    sort=nlp,
    description={Natural Language Processing (NLP) è un campo dell'intelligenza artificiale che si occupa di interazioni tra computer e linguaggio umano. L'obiettivo principale di NLP è consentire ai computer di comprendere, interpretare e generare il linguaggio umano in modo che possano effettivamente comunicare con gli esseri umani in modo naturale}
}

\newglossaryentry{ocrg} {
    name=\glslink{ocr}{OCR},
    text=OCR,
    sort = ocr,
    description = {Optical Character Recognition (OCR) è
    una tecnologia che permette di convertire diversi tipi di documenti cartacei o digitali in testo digitale, in modo che possano essere elaborati e analizzati da un computer}
}

\newglossaryentry{pecg} {
    name=\glslink{pec}{PEC},
    text=PEC,
    sort=pec,
    description={La \emph{Posta Elettronica Certificata} (PEC) è un servizio di posta elettronica che garantisce l'invio e la ricezione di messaggi di posta elettronica con valore legale equivalente a quello della raccomandata con avviso di ricevimento}
}

\newglossaryentry{repositoryg} {
    name=\glslink{repositoryg}{Repository},
    text = repository,
    sort=repository,
    description = {Con il termine repository si intende un ambiente di archiviazione centralizzato in cui vengono conservati e gestiti i file di un progetto software. Il repository consente di tenere traccia delle modifiche apportate ai file, di collaborare con altri sviluppatori e di mantenere una cronologia delle versioni del software
    }
}

\newglossaryentry{scalabilitàg}{
    name=\glslink{scalabilitàg}{Scalabilità},
    text=Scalabilità,
    sort=scalabilità,
    description={In informatica, la scalabilità è la capacità di un sistema di crescere in dimensioni e complessità in modo lineare o sub-lineare rispetto all'aumento del carico di lavoro}
}

\newglossaryentry{sdkg} {
    name=\glslink{sdk}{SDK},
    text=SDK,
    sort=sdk,
    description={A software development kit (SDK) is a collection of software development tools in one installable package. They facilitate the creation of applications by having a compiler, debugger and sometimes a software framework. They are normally specific to a hardware platform and operating system combination. To create applications with advanced functionalities such as advertisements, push notifications, etc; most application software developers use specific software development kits}
}

\newglossaryentry{scrumg}{
    name=\glslink{scrumg}{Scrum},
    text=Scrum,
    sort=scrum,
    description={In ingegneria del software, per Scrum si intende un framework agile per la gestione del ciclo di sviluppo del software. Scrum è caratterizzato da un approccio iterativo e incrementale, in cui il lavoro è organizzato in sprints di durata fissa, di solito di 2-4 settimane. Scrum prevede un team auto-organizzato e interfunzionale, che lavora in modo collaborativo per raggiungere gli obiettivi prefissati}
}

\newglossaryentry{umlg} {
    name=\glslink{uml}{UML},
    text=UML,
    sort=uml,
    description={In ingegneria del software \emph{Unified Modeling Language} (ing. linguaggio di modellazione unificato) è un linguaggio di modellazione e specifica basato sul paradigma object-oriented. L'\emph{UML} svolge un'importantissima funzione di ``lingua franca'' nella comunità della progettazione e programmazione a oggetti. Gran parte della letteratura di settore usa tale linguaggio per descrivere soluzioni analitiche e progettuali in modo sintetico e comprensibile a un vasto pubblico}
}

\newglossaryentry{serverlessg}{
    name=\glslink{serverlessg}{Serverless},
    text=serverless,
    sort=serverless,
    description={Per serverless si intende un modello di cloud computing in cui il fornitore di servizi cloud gestisce l'infrastruttura del server e le risorse di calcolo, e il cliente paga solo per il tempo di esecuzione delle funzioni. Il modello serverless consente di ridurre i costi e semplificare la gestione delle risorse, in quanto il cliente non deve preoccuparsi di configurare e mantenere i server}
}

\newglossaryentry{biasg}{
    name=\glslink{biasg}{Bias},
    text=bias,
    sort=bias,
    description={Nell'ambito del machine learning per bias si intende il fenomeno per cui un modello di machine learning è incline a fare previsioni errate a causa di dati di addestramento non rappresentativi o di un'architettura del modello sbagliata. Il bias può portare a discriminazioni e disuguaglianze, e può essere ridotto attraverso la raccolta di dati più rappresentativi e l'ottimizzazione dell'architettura del modello}
}

\newglossaryentry{activelearningg}{
    name=\glslink{activelearningg}{Active learning},
    text=active learning,
    sort=active learning,
    description={Nell'ambito del machine learning per active learning si intende una tecnica di apprendimento supervisionato in cui il modello di machine learning è in grado di selezionare autonomamente i campioni di addestramento più informativi da un pool di dati non etichettati. L'active learning consente di ridurre il costo dell'etichettatura dei dati e di migliorare le prestazioni del modello}
}

\newglossaryentry{bucketg}{
    name=\glslink{bucketg}{Bucket},
    text=bucket,
    sort=bucket,
    description={Nel contesto di AWS, per bucket si intende un contenitore di oggetti che consente di archiviare e organizzare i dati in Amazon S3. Un bucket può contenere un numero illimitato di oggetti e può essere configurato con diverse opzioni di accesso e sicurezza}
}

%\newglossaryentry{qiskitg} {
%name=\glslink{qiskit}{Qiskit},
%text=Qiskit,
%sort=qiskit,
%description={Qiskit ([quiss-kit] noun, software) %is an open-source SDK for working with quantum %computers at the level of extended quantum %circuits, operators, and primitives.}
%}
