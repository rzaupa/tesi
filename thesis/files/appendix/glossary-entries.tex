% Acronyms
\newacronym[description={\glslink{aig}{Artificial Intelligence}}]
{ai}{AI}{Artificial Intelligence}

\newacronym[description={\glslink{apig}{Application Programming Interface}}]
{api}{API}{Application Program Interface}

\newacronym[description={\glslink{awsg}{Amazon Web Services}}]
{aws}{AWS}{Amazon Web Services}

\newacronym[description={\glslink{erpg}{Enterprise Resource Planning}}]
{erp}{ERP}{Enterprise Resource Planning}

\newacronym[description={\glslink{ictg}{Information and Communication Technology}}]
{ict}{ICT}{Information and Communication Technology}

\newacronym[description={\glslink{idpg}{Intelligence Document Processing}}]
{idp}{IDP}{Intelligence Document Processing}

\newacronym[description={\glslink{mlg}{Machine Learning}}]
{ml}{ML}{Machine Learning}

\newacronym[description={\glslink{nlpg}{Natural Language Processing}}]
{nlp}{NLP}{Natural Language Processing}

\newacronym[description={\glslink{ocrg}{Optical Character Recognition}}]
{ocr}{OCR}{Optical Character Recognition}

\newacronym[description={\glslink{pecg}{Posta Elettronica Certificata}}]
{pec}{PEC}{Posta Elettronica Certificata}

\newacronym[description={\glslink{sdkg}{Software Development Kit}}]
{sdk}{SDK}{Software Development Kit}

\newacronym[description={\glslink{umlg}{Unified Modeling Language}}]
{uml}{UML}{Unified Modeling Language}

\newacronym[description={\glslink{bpmg}{Business Process Management}}]
{bpm}{BPM}{Business Process Management}

\newacronym[description={\glslink{cpqg}{Configure, Price, Quote}}]
{cpq}{CPQ}{Configure, Price, Quote}

\newacronym[description={\glslink{rpag}{Robotic Process Automation}}]
{rpa}{RPA}{Robotic Process Automation}

\newacronym[description={\glslink{llmg}{Large Language Model}}]
{llm}{LLM}{Large Language Model}

\newacronym[description={\glslink{mlopsg}{Machine Learning Operations}}]
{mlops}{MLOps}{Machine Learning Operations}

\newacronym[description={\glslink{ideg}{Integrated Development Environment}}]
{ide}{IDE}{Integrated Development Environment}

\newacronym[description={\glslink{fmg}{Foundation Models}}]
{fm}{FM}{Foundation Models}

\newacronym[description={\glslink{versioncontrolsystemg}{Version Control System}}]
{vcs}{VCS}{Version Control System}

% Glossary entries
\newglossaryentry{agileg}{
    name=\glslink{agileg}{Agile},
    text = Agile,
    sort = agile,
    description ={Nell'ambito dell'ingegneria del software con il termine Agile si intende un insieme di metodi di sviluppo del software basati su processi iterativi e incrementali, dove i requisiti e le soluzioni si evolvono attraverso la collaborazione tra team auto-organizzati e interfunzionali}
}

\newglossaryentry{aig}{
    name=\glslink{ai}{AI},
    text = AI,
    sort = ai,
    description = {Per artificial Intelligence (AI) si intende l'insieme di tecnologie e metodi che permettono ai computer di eseguire attività che richiedono intelligenza umana, come il riconoscimento di immagini, il riconoscimento vocale, la traduzione automatica, ecc}
}
% da mettere :
% idp e scalabilità
% Glossary entries
\newglossaryentry{apig} {
    name=\glslink{api}{API},
    text=API,
    sort=api,
    description={In informatica con il termine \emph{API} si indica ogni insieme di procedure disponibili al programmatore, di solito raggruppate a formare un set di strumenti specifici per l'espletamento di un determinato compito all'interno di un certo programma. La finalità è ottenere un'astrazione, di solito tra l'hardware e il programmatore o tra software a basso e quello ad alto livello semplificando così il lavoro di programmazione}
}

\newglossaryentry{awsg} {
    name=\glslink{aws}{AWS},
    text=AWS,
    sort=aws,
    description={Amazon Web Services (AWS) è una piattaforma di servizi cloud che offre potenza di calcolo, storage di database, distribuzione di contenuti e altre funzionalità per aiutare le imprese a scalare e crescere}
}

\newglossaryentry{ictg} {
    name=\glslink{ict}{ICT},
    text=ICT,
    sort=ict,
    description={Con il termine Information and Communication Technology (ICT) si intende l'insieme delle tecnologie informatiche e telematiche utilizzate per la gestione delle informazioni e la comunicazione}
}

\newglossaryentry{idpg}{
    name=\glslink{idp}{IDP},
    text=IDP,
    sort=idp,
    description={Con il termine Intelligence document processing (IDP) si intende l'insieme di tecnologie che permettono di estrarre informazioni da documenti cartacei o digitali, elaborarle e trasformarle in dati strutturati}
}

\newglossaryentry{mlg}{
    name=\glslink{ml}{ML},
    text=ML,
    sort=ml,
    description={Per Machine Learning (ML) si intende un insieme di tecniche e algoritmi che permettono ai computer di apprendere dai dati e di migliorare le prestazioni in base all'esperienza, senza essere esplicitamente programmati. Il machine learning si basa su modelli statistici e matematici che permettono di fare previsioni o decisioni in base ai dati analizzati}
}

\newglossaryentry{nlpg}{
    name=\glslink{nlp}{NLP},
    text=NLP,
    sort=nlp,
    description={Natural Language Processing (NLP) è un campo dell'intelligenza artificiale che si occupa di interazioni tra computer e linguaggio umano. L'obiettivo principale di NLP è consentire ai computer di comprendere, interpretare e generare il linguaggio umano in modo che possano effettivamente comunicare con gli esseri umani in modo naturale}
}

\newglossaryentry{ocrg} {
    name=\glslink{ocr}{OCR},
    text=OCR,
    sort = ocr,
    description = {Optical Character Recognition (OCR) è
    una tecnologia che permette di convertire diversi tipi di documenti cartacei o digitali in testo digitale, in modo che possano essere elaborati e analizzati da un computer}
}

\newglossaryentry{pecg} {
    name=\glslink{pec}{PEC},
    text=PEC,
    sort=pec,
    description={La \emph{Posta Elettronica Certificata} (PEC) è un servizio di posta elettronica che garantisce l'invio e la ricezione di messaggi di posta elettronica con valore legale equivalente a quello della raccomandata con avviso di ricevimento}
}

\newglossaryentry{repositoryg} {
    name=\glslink{repositoryg}{Repository},
    text = repository,
    sort=repository,
    description = {Con il termine repository si intende un ambiente di archiviazione centralizzato in cui vengono conservati e gestiti i file di un progetto software. Il repository consente di tenere traccia delle modifiche apportate ai file, di collaborare con altri sviluppatori e di mantenere una cronologia delle versioni del software
    }
}

\newglossaryentry{scalabilitàg}{
    name=\glslink{scalabilitàg}{Scalabilità},
    text=Scalabilità,
    sort=scalabilità,
    description={In informatica, la scalabilità è la capacità di un sistema di crescere in dimensioni e complessità in modo lineare o sub-lineare rispetto all'aumento del carico di lavoro}
}

\newglossaryentry{sdkg} {
    name=\glslink{sdk}{SDK},
    text=SDK,
    sort=sdk,
    description={Un Software Development Kit (SDK) è un insieme di strumenti e librerie di sviluppo software che consentono ai programmatori di creare applicazioni per una piattaforma specifica, come un sistema operativo, un framework o un servizio cloud}
}

\newglossaryentry{scrumg}{
    name=\glslink{scrumg}{Scrum},
    text=Scrum,
    sort=scrum,
    description={In ingegneria del software, per Scrum si intende un framework agile per la gestione del ciclo di sviluppo del software. Scrum è caratterizzato da un approccio iterativo e incrementale, in cui il lavoro è organizzato in sprints di durata fissa, di solito di 2-4 settimane. Scrum prevede un team auto-organizzato e interfunzionale, che lavora in modo collaborativo per raggiungere gli obiettivi prefissati}
}

\newglossaryentry{umlg} {
    name=\glslink{uml}{UML},
    text=UML,
    sort=uml,
    description={In ingegneria del software \emph{Unified Modeling Language} (ing. linguaggio di modellazione unificato) è un linguaggio di modellazione e specifica basato sul paradigma object-oriented. L'\emph{UML} svolge un'importantissima funzione di ``lingua franca'' nella comunità della progettazione e programmazione a oggetti. Gran parte della letteratura di settore usa tale linguaggio per descrivere soluzioni analitiche e progettuali in modo sintetico e comprensibile a un vasto pubblico}
}

\newglossaryentry{serverlessg}{
    name=\glslink{serverlessg}{Serverless},
    text=serverless,
    sort=serverless,
    description={Per serverless si intende un modello di cloud computing in cui il fornitore di servizi cloud gestisce l'infrastruttura del server e le risorse di calcolo, e il cliente paga solo per il tempo di esecuzione delle funzioni. Il modello serverless consente di ridurre i costi e semplificare la gestione delle risorse, in quanto il cliente non deve preoccuparsi di configurare e mantenere i server}
}

\newglossaryentry{biasg}{
    name=\glslink{biasg}{Bias},
    text=bias,
    sort=bias,
    description={Nell'ambito del machine learning per bias si intende il fenomeno per cui un modello di machine learning è incline a fare previsioni errate a causa di dati di addestramento non rappresentativi o di un'architettura del modello sbagliata. Il bias può portare a discriminazioni e disuguaglianze, e può essere ridotto attraverso la raccolta di dati più rappresentativi e l'ottimizzazione dell'architettura del modello}
}

\newglossaryentry{activelearningg}{
    name=\glslink{activelearningg}{Active learning},
    text=active learning,
    sort=active learning,
    description={Nell'ambito del machine learning per active learning si intende una tecnica di apprendimento supervisionato in cui il modello di machine learning è in grado di selezionare autonomamente i campioni di addestramento più informativi da un pool di dati non etichettati. L'active learning consente di ridurre il costo dell'etichettatura dei dati e di migliorare le prestazioni del modello}
}

\newglossaryentry{bucketg}{
    name=\glslink{bucketg}{Bucket},
    text=bucket,
    sort=bucket,
    description={Nel contesto di AWS, per bucket si intende un contenitore di oggetti che consente di archiviare e organizzare i dati in Amazon S3. Un bucket può contenere un numero illimitato di oggetti e può essere configurato con diverse opzioni di accesso e sicurezza}
}

\newglossaryentry{erpg}{
    name=\glslink{erpg}{ERP},
    text=ERP,
    sort=erp,
    description={Enterprise Resource Planning (ERP) è un sistema di gestione aziendale che integra e automatizza i processi aziendali, come la contabilità, la gestione delle risorse umane, la produzione, la logistica, le vendite e il marketing. Un sistema ERP consente di migliorare l'efficienza, la produttività e la collaborazione all'interno dell'azienda}
}

\newglossaryentry{dataprotectiong}{
    name=\glslink{dataprotectiong}{Data protection},
    text=Data Protection,
    sort=data protection,
    description={Misure tecniche e organizzative che proteggono i dati personali e aziendali da accessi non autorizzati, modifiche, divulgazioni o distruzioni.}
}

\newglossaryentry{cybersecurityg}{
    name=\glslink{cybersecurityg}{Cybersecurity},
    text=Cybersecurity,
    sort=cybersecurity,
    description={La cybersecurity è l'insieme di pratiche e tecnologie utilizzate per proteggere sistemi, reti e dati da attacchi informatici, accessi non autorizzati, e altre minacce digitali.}
}

\newglossaryentry{ibmpowerg}{
    name=\glslink{ibmpowerg}{IBM Power},
    text=IBM Power,
    sort=ibm power,
    description={IBM Power Systems sono una famiglia di server e processori sviluppati da IBM utilizzati in ambienti aziendali per applicazioni critiche}
}

\newglossaryentry{gdpr}{
    name=\glslink{gdpr}{GDPR},
    text=GDPR,
    sort=gdpr,
    description={Il Regolamento Generale sulla Protezione dei Dati (GDPR) è una legge sulla privacy che regola la protezione dei dati personali dei cittadini dell'Unione Europea. Il GDPR è entrato in vigore il 25 maggio 2018 e stabilisce regole chiare per la raccolta, l'elaborazione e la conservazione dei dati personali}
}

\newglossaryentry{businessintelligenceg}{
    name=\glslink{businessintelligenceg}{Business Intelligence},
    text=Business Intelligence,
    sort=business intelligence,
    description={La Business Intelligence (BI) è un insieme di processi, strumenti e tecnologie che consentono alle aziende di raccogliere, analizzare e presentare informazioni aziendali per supportare la presa di decisioni informate. La BI si basa sull'analisi dei dati storici e in tempo reale per identificare tendenze, modelli e opportunità di business}
}

\newglossaryentry{performancemanagementg}{
    name=\glslink{performancemanagementg}{Performance Management},
    text=Performance Management,
    sort=performance management,
    description={Il Performance Management è un processo continuo di pianificazione, monitoraggio e valutazione delle prestazioni dei dipendenti per garantire che raggiungano gli obiettivi aziendali. Il Performance Management coinvolge la definizione degli obiettivi, la valutazione delle prestazioni, il feedback e lo sviluppo delle competenze}
}

\newglossaryentry{customerintelligenceg}{
    name=\glslink{customerintelligenceg}{Customer Intelligence},
    text=Customer Intelligence,
    sort=customer intelligence,
    description={La Customer Intelligence è l'insieme di processi, strumenti e tecnologie che consentono alle aziende di raccogliere, analizzare e utilizzare informazioni sui clienti per migliorare la customer experience, aumentare la fedeltà dei clienti e massimizzare il valore del cliente}
}

\newglossaryentry{governanceaziendaleg}
{
    name=\glslink{governanceaziendaleg}{Governance aziendale},
    text=Governance aziendale,
    sort=governance aziendale,
    description={La governance aziendale è il sistema di regole, processi e pratiche che guidano e controllano le attività e le decisioni all'interno di un'azienda. La governance aziendale si occupa di definire gli obiettivi, le politiche e le procedure dell'azienda, di monitorare le prestazioni e di garantire la conformità alle normative e agli standard}
}

\newglossaryentry{bpmg}{
    name=\glslink{bpmg}{BPM},
    text=BPM,
    sort=bpm,
    description={Il Business Process Management (BPM) è un approccio sistemico alla gestione dei processi aziendali che mira a migliorare l'efficienza, la qualità e l'agilità dell'azienda. Il BPM coinvolge l'analisi, la progettazione, l'automazione e il monitoraggio dei processi aziendali per garantire che siano allineati agli obiettivi aziendali e alle esigenze dei clienti}
}

\newglossaryentry{cpqg}{
    name=\glslink{cpqg}{CPQ},
    text=CPQ,
    sort=cpq,
    description={Il Configure, Price, Quote (CPQ) è un processo aziendale che consente alle aziende di configurare, quotare e vendere prodotti e servizi in modo rapido, accurato e redditizio. Il CPQ coinvolge la configurazione dei prodotti, la determinazione dei prezzi e la generazione di preventivi personalizzati per i clienti}
}

\newglossaryentry{rpag}{
    name=\glslink{rpag}{RPA},
    text=RPA,
    sort=rpa,
    description={La Robotic Process Automation (RPA) è una tecnologia che consente di automatizzare i processi aziendali ripetitivi e basati su regole utilizzando software robot. I robot software possono eseguire attività manuali, ripetitive e noiose in modo rapido, accurato e senza errori}
}

\newglossaryentry{llmg}{
    name=\glslink{llm}{LLM},
    text=LLM,
    sort=llm,
    description={Un Large Language Model (LLM) è un modello di linguaggio basato su reti neurali profonde che è stato addestrato su un vasto corpus di testo per generare testo naturale in modo coerente e convincente. Gli LLM sono utilizzati in applicazioni di generazione di testo, traduzione automatica, riassunto automatico e altre applicazioni di elaborazione del linguaggio naturale}
}

\newglossaryentry{generativeaig}{
    name=\glslink{generativeaig}{Generative AI},
    text=Generative AI,
    sort=generative ai,
    description={La Generative AI è un'area dell'intelligenza artificiale che si occupa di creare nuovi contenuti, come immagini, testo, musica e video, utilizzando modelli di machine learning generativi. La Generative AI è utilizzata in applicazioni di creazione di contenuti, design assistito da computer, e generazione di arte e musica}
}

\newglossaryentry{mlopsg}{
    name=\glslink{mlopsg}{MLOps},
    text=MLOps,
    sort=mlops,
    description={MLOps è una pratica che combina i principi e le pratiche dell'ingegneria del software con quelli del machine learning per creare, implementare e gestire modelli di machine learning in modo efficiente ed efficace. MLOps coinvolge la collaborazione tra team di sviluppo, data science e operazioni per garantire che i modelli di machine learning siano scalabili, affidabili e sicuri}
}

\newglossaryentry{deeplearningg}{
    name=\glslink{deeplearningg}{Deep Learning},
    text=Deep Learning,
    sort=deep learning,
    description={Il Deep Learning è un'area dell'intelligenza artificiale che si occupa di creare modelli di machine learning basati su reti neurali profonde. Il Deep Learning è utilizzato in applicazioni di riconoscimento di immagini, riconoscimento vocale, traduzione automatica e altre applicazioni di elaborazione del linguaggio naturale}
}

\newglossaryentry{computervisiong}
{
    name=\glslink{computervisiong}{Computer Vision},
    text=Computer Vision,
    sort=computer vision,
    description={La Computer Vision è un'area dell'intelligenza artificiale che si occupa di creare sistemi che possono interpretare e comprendere le immagini e i video in modo simile agli esseri umani. La Computer Vision è utilizzata in applicazioni di riconoscimento facciale, riconoscimento di oggetti, veicoli autonomi e altre applicazioni di visione artificiale}
}

\newglossaryentry{nosqlg}{
    name=\glslink{nosqlg}{NoSQL},
    text=NoSQL,
    sort=nosql,
    description={Il NoSQL è un'approccio alla gestione dei dati che si basa su modelli di dati non relazionali, come i database di documenti, i database di colonne, i database di grafi e i database chiave-valore. Il NoSQL è utilizzato per gestire grandi volumi di dati non strutturati e semi-strutturati in modo flessibile ed efficiente}
}

\newglossaryentry{ideg}{
    name=\glslink{ideg}{IDE},
    text=IDE,
    sort=ide,
    description={Un Integrated Development Environment (IDE) è un software che fornisce un ambiente integrato per lo sviluppo di software, comprensivo di editor di codice, compilatore, debugger e altre funzionalità di sviluppo. Un IDE semplifica il processo di sviluppo del software e aumenta la produttività dei programmatori}
}

\newglossaryentry{fmg}{
    name=\glslink{fm}{FM},
    text=FM,
    sort=fm,
    description={I Foundation Models (FM) sono modelli di linguaggio basati su reti neurali profonde che sono stati addestrati su un vasto corpus di testo per generare testo naturale in modo coerente e convincente. I FM sono utilizzati in applicazioni di generazione di testo, traduzione automatica, riassunto automatico e altre applicazioni di elaborazione del linguaggio naturale}
}

\newglossaryentry{versioncontrolsystemg}{
    name=\glslink{versioncontrolsystemg}{Version Control System},
    text=Version Control System,
    sort=version control system,
    description={Un Version Control System (VCS) è un sistema che registra le modifiche apportate ai file di un progetto software nel tempo, in modo che sia possibile ripristinare versioni precedenti, confrontare le modifiche e collaborare con altri sviluppatori. I VCS sono utilizzati per tenere traccia delle modifiche al codice sorgente e coordinare il lavoro di sviluppo}
}
%\newglossaryentry{qiskitg} {
%name=\glslink{qiskit}{Qiskit},
%text=Qiskit,
%sort=qiskit,
%description={Qiskit ([quiss-kit] noun, software) %is an open-source SDK for working with quantum %computers at the level of extended quantum %circuits, operators, and primitives.}
%}
