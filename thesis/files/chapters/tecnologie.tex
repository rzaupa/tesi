\chapter{Tecnologie e strumenti di interesse}
\label{cap:tecnologie}
\emph{In questo capitolo verranno descritti i servizi e le tecnologie analizzate e pertinenti per il problema descritto, in quale modo possono essere impiegate e una panoramica finalizzata a chiarirne il constesto e il caso d'uso.}

\section{Amazon Web Services}
\gls{aws} è una piattaforma di servizi cloud che offre potenza di calcolo, storage di database, distribuzione di contenuti e altre funzionalità per aiutare le aziende a scalare e crescere. AWS offre una vasta gamma di servizi che possono essere utilizzati per implementare soluzioni di \gls{ai} e \glsfirstoccur{\gls{ml}}. Per la realizzazione dell'applicazione sono stati individuati diversi servizi che hanno permesso di realizzare un'archittettura scalabile e \gls{serverlessg}.

\subsection{Amazon Comprehend}
Amazon Comprehend (logo in \ref{fig:comprehend}) è un servizio di analisi del linguaggio naturale \gls{nlpg} che utilizza l'apprendimento automatico per identificare informazioni e connessioni utili nel testo. Tale servizio offre due funzionalità principali: Custom Classifier e Entity Classifier. 
Entrambi i servizi permettono di generare dei modelli dopo la fase di training creata mediante dei file csv (label e testo).\\
Un servizio di Comprehend utile in questo contesto è Flywheel che è il punto di riferimento principale per eseguire MLOps (Machine Learning Operations) per i modelli di Comprehend. \\
Amazon Comprehend è stato utilizzato per classificare i documenti nelle categorie selezionate. 

\begin{figure}[h]
  \centering
  \includegraphics[width=0.2\textwidth]{img/tecnologie/comprehend.png}
  \caption{Logo di Amazon Comprehend}
  \label{fig:comprehend}
\end{figure}

\subsection{Amazon Textract}
Amazon Textract (logo in \ref{fig:textract}) è un servizio di \gls{ocr} che utilizza l'apprendimento automatico per riconoscere e analizzare il testo e i dati da immagini o documenti. \\
Tale servizio oltre ad effettuare l'identificazione ottica dei caratteri permette di identificare il contenuto del documento estraendone il testo, le tabelle, campi e relazioni. Oltre al contenuto rilevato, Amazon Textract, fornisce punteggi di confidenza e bounded box (box di confine) per ogni parola e ogni riga di testo.
Supporta i file pdf, txt, doc, docx, jpg, png.\\
I casi d'uso che Textract ricopre sono i seguenti:
\begin{itemize}
  \item Estrazione del testo non strutturato (DetectDocumentText). Questa funzionalità permette di estrarre i dati in forma di WORDS e LINES perdendo la formattazione struturale del documento. 
  \item Estrazione ed elaborazione di moduli e tabelle (AnalyzeDocument, feature$=$TABLES)
  \item Estrazione di form (chiave/valore) (AnalyzeDocument, feature$=$FORMS)
  \item Estrazione tramite query. Possono essere degli strumenti potenti quando solo pochi pezzi o informazioni critiche sono desiderate. (AnalyzeDocument, feature$=$QUERIES). 
  \item Rilevamento della firma. (AnalyzeDocument, feature$=$SIGNATURES). La risposta prevede la confidenza della rilevazione e il testo del documento (WORDS e LINES). 
  \item Estrazione informazioni da fatture/ricevute (AnalyzeExpense)
  \item Estrazione informazioni dai documenti di identità (CALL\_TEXTRACT\_ANALYZE\_ID)
  \item Rilevamento multicolonna
\end{itemize}
Per personalizzare le query si può usare uno strumento chiamato Custom Queries. Tramite questo strumento si può riconoscere:
\begin{itemize}
  \item termini univoci
  \item strutture
  \item informazioni specifiche
\end{itemize}
Tale soluzione rispetto alle query non personalizzate offre maggiore precisione e un intervento umano minore. 
Amazon Textract è stato utilizzato per estrarre il testo dai documenti sia come input al classificatore di Comprehend sia per estrarre informazioni utili.

\begin{figure}[h]
  \centering
  \includegraphics[width=0.2\textwidth]{img/tecnologie/textract.png}
  \caption{Logo di Amazon Textract}
  \label{fig:textract}
\end{figure}

\subsection{Amazon S3}
Amazon Simple Storage Service (Amazon S3) è un servizio di storage di oggetti che offre scalabilità, disponibilità dei dati, sicurezza e prestazioni. Amazon S3 è progettato per memorizzare grandi quantità di dati a un costo molto basso. Amazon S3 è stato utilizzato per memorizzare i file inerenti alle diversi fasi del progetto.

\begin{figure}[h]
  \centering
  \includegraphics[width=0.2\textwidth]{img/tecnologie/s3.png}
  \caption{Logo di Amazon S3}
  \label{fig:s3}
\end{figure}

\subsection{AWS Lambda}
AWS Lambda è un servizio di calcolo \gls{serverlessg} che esegue il codice in risposta a eventi e gestisce automaticamente le risorse di calcolo richieste dal codice. AWS Lambda è stato utilizzato per implementare le funzioni di backend dell'applicazione. 

\begin{figure}[h]
  \centering
  \includegraphics[width=0.3\textwidth]{img/tecnologie/AWS_Lambda.png}
  \caption{Logo di AWS Lambda}
  \label{fig:lambda}
\end{figure}

\subsection{Amazon DynamoDB}
Amazon DynamoDB è un servizio di database NoSQL completamente gestito che offre prestazioni di singolo millisecondo a qualsiasi scala. Amazon DynamoDB è stato utilizzato per memorizzare i dati relativi ai vari utenti dell'applicazione.

\begin{figure}[h]
  \centering
  \includegraphics[width=0.3\textwidth]{img/tecnologie/DynamoDB.png}
  \caption{Logo di Amazon DynamoDB}
  \label{fig:dynamodb}
\end{figure}

\subsection{Amazon Step Functions}
AWS Step Functions è un servizio di orchestrazione di \gls{serverlessg} che consente di coordinare facilmente i componenti di applicazioni distribuite e microservizi utilizzando logica visuale. 

\begin{figure}[h]
  \centering
  \includegraphics[width=0.3\textwidth]{img/tecnologie/stepfunctions.png}
  \caption{Logo di Amazon Step Functions}
  \label{fig:stepfunctions}
\end{figure}

\subsection{Amazon Sagemaker}

\subsection{Amazon Bedrock}

\section{Strumenti di sviluppo}
\subsection{Visual Studio Code}
Visual Studio Code è un editor di codice sorgente sviluppato da Microsoft per Windows, Linux e macOS. Visual Studio Code è stato utilizzato per scrivere il codice dell'applicazione.

\begin{figure}[h]
  \centering
  \includegraphics[width=0.3\textwidth]{img/tecnologie/vscode.png}
  \caption{Logo di Visual Studio Code}
  \label{fig:vscode}
\end{figure}

\subsection{Git}
\glsfirstoccur{\gls{gitg}} è un sistema di controllo di versione distribuito utilizzato per tenere traccia delle modifiche al codice sorgente durante lo sviluppo del software. Git è stato utilizzato per tenere traccia delle modifiche al codice sorgente dell'applicazione.

\begin{figure}[h]
  \centering
  \includegraphics[width=0.3\textwidth]{img/tecnologie/git.png}
  \caption{Logo di Git}
  \label{fig:git}
\end{figure}

\subsection{Bitbucket}
Bitbucket è un servizio di hosting di \glsfirstoccur{\gls{repositoryg}} Git basato su cloud. Bitbucket è stato utilizzato per memorizzare il codice sorgente dell'applicazione.

\begin{figure}[h]
  \centering
  \includegraphics[width=0.5\textwidth]{img/tecnologie/bitbucket.png}
  \caption{Logo di Bitbucket}
  \label{fig:bitbucket}
\end{figure}

\section{Linguaggi di programmazione}
\subsection{Python}
Python è un linguaggio di programmazione ad alto livello, interpretato, adatto per lo sviluppo di applicazioni web, desktop e mobile. Python è stato utilizzato per la realizzazione del backend dell'applicazione.

\begin{figure}[h]
  \centering
  \includegraphics[width=0.2\textwidth]{img/tecnologie/python.png}
  \caption{Logo di Python}
  \label{fig:python}
\end{figure}