\chapter{Tecnologie e strumenti di interesse}
\label{cap:tecnologie}
\emph{In questo capitolo verranno descritti i servizi e le tecnologie analizzate e pertinenti per il problema descritto, in quale modo possono essere impiegate e una panoramica finalizzata a chiarirne il contesto e il caso d'uso.}

\section{Amazon Web Services}
\gls{aws} è una piattaforma di servizi cloud che offre potenza di calcolo, storage di database, distribuzione di contenuti e altre funzionalità per aiutare le aziende a scalare e crescere. AWS offre una vasta gamma di servizi che possono essere utilizzati per implementare soluzioni di \gls{ai} e \glsfirstoccur{\gls{ml}}. Per la realizzazione dell'applicazione sono stati individuati diversi servizi che hanno permesso di realizzare un'architettura scalabile e \gls{serverlessg}.

\subsection{Amazon Comprehend}
Amazon Comprehend (il logo è riportato in Figura \ref{fig:comprehend}) è un servizio avanzato di analisi del linguaggio naturale (\gls{nlpg}) che utilizza algoritmi di apprendimento automatico per estrarre informazioni significative dai testi. Il servizio è in grado di identificare entità, frasi chiave, lingua, sentimenti e altre caratteristiche comuni all'interno dei documenti, offrendo la possibilità di effettuare analisi sia in tempo reale che in modalità asincrona su grandi volumi di dati. Gli utenti possono scegliere di utilizzare modelli pre-addestrati o di addestrare modelli personalizzati per specifiche esigenze di classificazione e riconoscimento delle entità.\\
Tra le principali funzionalità di Amazon Comprehend vi è \textit{Amazon Comprehend Insights}, che consente di analizzare documenti, singoli o in gruppo, per identificare le informazioni più rilevanti utilizzando modelli già addestrati. Questi modelli possono essere impiegati per individuare entità (come persone, luoghi, date, quantità, ecc.), frasi chiave, informazioni personali identificabili (PII, \textit{Personally Identifiable Information}), sentimenti (positivo, negativo, neutro, misto), oltre a determinare la lingua e la sintassi del testo.\\
Un'altra funzionalità rilevante è \textit{Amazon Comprehend Custom}, che permette la creazione di modelli NLP personalizzati per la classificazione (\textit{Custom Classification}) e il riconoscimento delle entità (\textit{Custom Entity Recognition}). La \textit{Custom Classification} consente di categorizzare i documenti in base a categorie predefinite, mentre la \textit{Custom Entity Recognition} permette di individuare entità specifiche all'interno dei testi. Entrambi i servizi richiedono una fase di training che necessita di un dataset etichettato per addestrare il modello e supportano le'elaborazione dei documenti in un'unica fase.\\
In aggiunta, Amazon Comprehend offre la funzionalità \textit{Flywheel}, che semplifica il processo di addestramento e gestione delle versioni dei modelli personalizzati, facilitando l'orchestrazione delle attività di training, valutazione e deployment dei modelli. Consiste dunque nel riferimento principale per la fase di MLOps (\textit{Machine Learning Operations}) e permette di monitorare le prestazioni dei modelli, valutare le metriche di accuratezza e precisione e gestire le versioni dei modelli in produzione.\\
Infine, il \textit{Document Clustering} permette di raggruppare i documenti in base a parole chiave ricorrenti, rendendo più agevole l'identificazione di documenti simili e la loro organizzazione per categorie o argomenti.\\
Nel presente lavoro, Amazon Comprehend è stato utilizzato per la classificazione dei documenti nelle categorie selezionate tramite la funzionalità \textit{Custom Classification}.

\begin{figure}[h]
  \centering
  \includegraphics[width=0.2\textwidth]{img/tecnologie/comprehend.png}
  \caption{Logo di Amazon Comprehend}
  \label{fig:comprehend}
\end{figure}

\subsection{Amazon Textract}
Amazon Textract (il logo è riportato in Figura \ref{fig:textract}) è un servizio di riconoscimento ottico dei caratteri (\gls{ocr}) che sfrutta l'apprendimento automatico per identificare e analizzare testo e dati presenti in immagini o documenti. Basato sulla tecnologia di deep learning collaudata e altamente scalabile sviluppata dagli esperti di visione artificiale di Amazon, Textract è in grado di analizzare quotidianamente miliardi di immagini e video. Una delle caratteristiche distintive di questo servizio è la sua accessibilità: non è richiesta alcuna esperienza nel campo del machine learning per utilizzarlo, grazie alla disponibilità di API semplici e intuitive che consentono di analizzare file immagine e PDF con facilità. Inoltre, Amazon Textract apprende continuamente dai nuovi dati e Amazon implementa costantemente nuove funzionalità, garantendo un miglioramento continuo delle sue capacità.

Il servizio non si limita a eseguire il riconoscimento ottico dei caratteri da testo digitato o scritto a mano, ma è anche in grado di estrarre il contenuto del documento, incluse tabelle, campi e relazioni strutturali. Textract fornisce punteggi di confidenza e bounding box (rappresentazioni grafiche dei confini) per ogni parola e riga di testo riconosciuta. Il servizio supporta vari formati di file, tra cui PDF, TXT, DOC, DOCX, JPG e PNG.

Le principali funzionalità di Amazon Textract includono:

\begin{itemize}
    \item \textbf{Estrazione di testo non strutturato}: Questa funzionalità consente di estrarre i dati in forma di parole (\textit{WORDS}) e righe di testo (\textit{LINES}), senza mantenere la formattazione originaria del documento. Per questa operazione si utilizza l'API \texttt{DetectDocumentText}.
    
    \item \textbf{Estrazione ed elaborazione di moduli e tabelle}: Tramite l'API \texttt{AnalyzeDocument}, è possibile estrarre dati mantenendo la struttura del documento originale, identificando parole, righe, tabelle e moduli (\textit{WORDS}, \textit{LINES}, \textit{TABLES}, \textit{FORMS}).
    
    \item \textbf{Estrazione di coppie chiave-valore}: Utilizzando l'API \texttt{AnalyzeDocument}, questa funzionalità permette di estrarre informazioni strutturate in forma di chiavi e valori, preservando la formattazione del documento.
    
    \item \textbf{Estrazione tramite query}: Questa funzionalità consente di focalizzarsi su informazioni specifiche o critiche all'interno di un documento. Anche in questo caso, l'API utilizzata è \texttt{AnalyzeDocument}.
    
    \item \textbf{Rilevamento delle firme}: Attraverso l'API \texttt{AnalyzeDocument}, è possibile rilevare la presenza di firme nei documenti, restituendo un punteggio di confidenza per il rilevamento, oltre al testo del documento in forma di parole e righe (\textit{WORDS} e \textit{LINES}).
    
    \item \textbf{Estrazione di informazioni da fatture e ricevute}: L'API \texttt{AnalyzeExpense} è specificamente progettata per estrarre dati da documenti contabili come fatture e ricevute.
    
    \item \textbf{Estrazione di informazioni da documenti di identità}: Utilizzando l'API \texttt{AnalyzeID}, è possibile estrarre dati rilevanti da documenti di identità.
    
    \item \textbf{Rilevamento di testo su più colonne}: Questa funzionalità consente di riconoscere e trattare testi distribuiti su più colonne all'interno di un documento.
\end{itemize}

Per migliorare la precisione delle analisi e ridurre l'intervento umano necessario, Amazon Textract offre lo strumento delle \textit{Custom Queries}. Questo strumento consente di riconoscere specifici termini univoci, strutture particolari e informazioni specifiche all'interno dei documenti, offrendo un livello di personalizzazione superiore rispetto alle query standard.

Un'altra opzione avanzata per personalizzare l'output dell'analisi dei documenti è l'uso degli \textit{Adapters}. Gli Adapters sono componenti che si integrano nel modello di deep learning pre-addestrato di Amazon Textract, permettendo di personalizzare l'output in base ai documenti specifici di un'azienda. Per creare un Adapter, è necessario annotare ed etichettare un insieme di documenti campione e addestrare l'Adapter su questi campioni annotati.

Una volta creato un Adapter, Amazon Textract fornisce un \textit{AdapterId}. È possibile creare e gestire diverse versioni di un Adapter all'interno di uno stesso identificatore. L'\textit{AdapterId}, insieme alla versione dell'Adapter, può essere utilizzato in una richiesta per specificare l'uso dell'Adapter creato durante l'analisi dei documenti. Ad esempio, questi parametri possono essere forniti all'API \texttt{AnalyzeDocument} per un'analisi sincrona dei documenti, oppure all'operazione \texttt{StartDocumentAnalysis} per un'analisi asincrona. Includendo l'\textit{AdapterId} nella richiesta, l'Adapter verrà automaticamente integrato nel processo di analisi, migliorando le previsioni per i documenti specifici.

Questo approccio consente di sfruttare le capacità dell'API \texttt{AnalyzeDocument} mentre si adatta il modello alle esigenze specifiche del proprio caso d'uso. 

Nel contesto del presente lavoro, Amazon Textract è stato utilizzato per estrarre il testo dai documenti sia come input al classificatore di Comprehend sia per estrarre informazioni utili.

\begin{figure}[h]
  \centering
  \includegraphics[width=0.2\textwidth]{img/tecnologie/textract.png}
  \caption{Logo di Amazon Textract}
  \label{fig:textract}
\end{figure}

\subsection{Amazon S3}
Amazon Simple Storage Service (Amazon S3) (logo riportato in Figura \ref{fig:s3}) è un servizio di storage di oggetti che offre elevata scalabilità, disponibilità dei dati, sicurezza e prestazioni. Amazon S3 è progettato per gestire grandi volumi di dati a costi contenuti, risultando una soluzione ideale per applicazioni che richiedono capacità di archiviazione massiva.

Per memorizzare dati in Amazon S3, è necessario utilizzare un \textit{bucket}, che funge da contenitore per gli oggetti. Ogni oggetto in un bucket rappresenta un file e i relativi metadati associati. La procedura per archiviare un oggetto in Amazon S3 prevede la creazione di un bucket e il successivo caricamento dell'oggetto al suo interno. Una volta caricato, l'oggetto può essere aperto, scaricato o eliminato. Qualora un oggetto o un bucket non siano più necessari, è possibile procedere alla loro eliminazione.

Nel contesto del presente progetto, Amazon S3 è stato utilizzato per memorizzare i file relativi alle diverse fasi del lavoro, inclusi allegati, email, file CSV impiegati per l'addestramento dei modelli e file di output generati dalle analisi. 


\begin{figure}[h]
  \centering
  \includegraphics[width=0.2\textwidth]{img/tecnologie/s3.png}
  \caption{Logo di Amazon S3}
  \label{fig:s3}
\end{figure}

\subsection{AWS Lambda}
AWS Lambda (logo riportato in Figura \ref{fig:lambda}) è un servizio di calcolo \gls{serverlessg} che esegue codice in risposta a eventi, gestendo automaticamente le risorse di calcolo necessarie. Questo servizio elimina la necessità di provisioning e gestione dei server, offrendo una soluzione scalabile e affidabile per diverse applicazioni.

Il codice in Lambda è organizzato in funzioni che vengono eseguite solo quando richiesto, scalando automaticamente in base al carico. La tariffazione si basa esclusivamente sul tempo di calcolo utilizzato, senza costi aggiuntivi quando il codice non è in esecuzione. Questa flessibilità lo rende ideale per scenari che richiedono scalabilità dinamica e riduzione automatica delle risorse in assenza di carico.

Nel contesto del presente progetto, AWS Lambda è stato impiegato per implementare le funzioni di chiamate API, garantendo un'architettura serverless efficiente. Le funzioni Lambda sono state integrate con altri servizi AWS, come Amazon S3 per l'elaborazione dei file e Amazon API Gateway per la gestione delle richieste API. L'adozione di Lambda ha permesso di semplificare la gestione operativa, poiché il servizio si occupa automaticamente di capacità, monitoraggio e logging, lasciando agli sviluppatori la responsabilità esclusiva del codice.


\begin{figure}[h]
  \centering
  \includegraphics[width=0.3\textwidth]{img/tecnologie/AWS_Lambda.png}
  \caption{Logo di AWS Lambda}
  \label{fig:lambda}
\end{figure}

\subsection{Amazon DynamoDB}
Amazon DynamoDB (logo riportato in Figura \ref{fig:dynamodb}) è un servizio di database NoSQL completamente gestito, progettato per garantire prestazioni a singola cifra di millisecondi indipendentemente dalla scala. Ideale per carichi di lavoro operativi che richiedono alta efficienza, DynamoDB affronta le complessità di scalabilità e gestione operativa tipiche dei database relazionali, mantenendo prestazioni elevate anche in presenza di un grande numero di utenti. Questo lo rende particolarmente adatto per applicazioni moderne che necessitano di crescere rapidamente a livello globale.

Dal suo lancio nel 2012, DynamoDB è stato adottato da organizzazioni di ogni settore e dimensione per sviluppare applicazioni che possono iniziare con piccoli volumi di dati e scalare fino a supportare tabelle di dimensioni virtualmente illimitate, assicurando al contempo alta disponibilità.

Nel contesto del presente progetto, Amazon DynamoDB è stato utilizzato per la memorizzazione dei dati estratti dai documenti e delle classificazioni effettuate, garantendo un accesso rapido e affidabile alle informazioni archiviate.



\begin{figure}[h]
  \centering
  \includegraphics[width=0.3\textwidth]{img/tecnologie/DynamoDB.png}
  \caption{Logo di Amazon DynamoDB}
  \label{fig:dynamodb}
\end{figure}
\subsection{AWS Step Functions}
AWS Step Functions (logo riportato in Figura \ref{fig:stepfunctions}) è un servizio di orchestrazione \gls{serverlessg} che consente di coordinare in modo efficiente i componenti di applicazioni distribuite, microservizi e pipeline di dati o di machine learning attraverso una logica visuale. Questo servizio si basa sul concetto di macchine a stati (\textit{State machines}) e task, dove una macchina a stati, o workflow, è costituita da una serie di passaggi guidati da eventi. Ogni passaggio nel workflow è chiamato stato, e uno stato di tipo Task rappresenta un'unità di lavoro eseguita da un altro servizio AWS o API. Le esecuzioni, ovvero le istanze di workflow in esecuzione, sono gestite direttamente da Step Functions.

Le attività all'interno dei task della macchina a stati possono anche essere svolte utilizzando le \textit{Activities}, che sono lavoratori esterni al servizio Step Functions.

Nel contesto del presente progetto, AWS Step Functions è stato utilizzato per orchestrare i vari servizi AWS coinvolti, in particolare le funzioni Lambda. 


\begin{figure}[h]
  \centering
  \includegraphics[width=0.3\textwidth]{img/tecnologie/stepfunctions.png}
  \caption{Logo di Amazon Step Functions}
  \label{fig:stepfunctions}
\end{figure}

\subsection{Amazon SageMaker}
Amazon SageMaker (logo riportato in Figura \ref{fig:sagemaker}) è un servizio completamente gestito per il \textit{machine learning} (ML) che permette a data scientist e sviluppatori di costruire, addestrare e distribuire modelli ML in un ambiente di produzione altamente scalabile e sicuro. SageMaker facilita l'intero processo di sviluppo di modelli ML, fornendo un'interfaccia utente intuitiva che integra strumenti e funzionalità di ML all'interno di diversi ambienti di sviluppo integrato (IDE).

SageMaker consente di archiviare e condividere i dati senza dover gestire infrastrutture server, permettendo alle organizzazioni di concentrarsi sullo sviluppo collaborativo dei flussi di lavoro ML. Il servizio supporta algoritmi ML gestiti, ottimizzati per elaborare grandi volumi di dati in un ambiente distribuito, e offre la flessibilità di utilizzare algoritmi e framework personalizzati. In pochi passaggi, è possibile distribuire un modello in un ambiente sicuro e scalabile direttamente dalla console di SageMaker.

Tra gli strumenti offerti da Amazon SageMaker vi sono:

\begin{itemize}
    \item \textbf{Amazon SageMaker JumpStart}: Un hub di ML che consente di valutare e selezionare modelli fondamentali (\textit{foundation models}) in base a specifici parametri.
    \item \textbf{Amazon SageMaker Studio}: Un IDE completo per preparare i dati, creare, addestrare e distribuire modelli ML, offrendo strumenti per ogni fase del ciclo di vita del ML.
    \item \textbf{Amazon SageMaker MLOps}: Fornisce strumenti per automatizzare le operazioni di ML lungo tutto il ciclo di vita del modello, inclusi processi di integrazione e distribuzione continua (CI/CD).
    \item \textbf{Amazon SageMaker BlazingText}: Implementa l'algoritmo Word2Vec per la creazione di vettori di parole, utilizzati nell'elaborazione del linguaggio naturale.
    \item \textbf{Pipeline di Amazon SageMaker}: Automatizza le diverse fasi del ML, dalla pre-elaborazione dei dati al monitoraggio dei modelli in produzione.
    \item \textbf{Amazon SageMaker Ground Truth}: Migliora la precisione dei modelli ML sfruttando il feedback umano durante tutto il ciclo di vita del modello, permettendo anche la creazione di etichette per i dati.
    \item \textbf{Amazon SageMaker Clarify}: Rileva e mitiga i pregiudizi presenti nei dati di addestramento e nelle previsioni dei modelli ML.
    \item \textbf{Amazon SageMaker Model Monitor}: Monitora i modelli ML in produzione per rilevare eventuali cambiamenti nei dati o nelle prestazioni dei modelli, assicurando un'accuratezza costante nel tempo.
\end{itemize}

Nel contesto del presente progetto, Amazon SageMaker non è stato utilizzato direttamente, in quanto si è ritenuto l'utilizzo di Amazon Comprehend e Amazon Textract sufficiente per le esigenze di analisi del testo e dei documenti. Tuttavia, SageMaker rappresenta una risorsa fondamentale per lo sviluppo di modelli ML personalizzati e per l'implementazione di soluzioni di ML avanzate.
\begin{figure}[h]
  \centering
  \includegraphics[width=0.2\textwidth]{img/tecnologie/sagemaker.png}
  \caption{Logo di Amazon SageMaker}
  \label{fig:sagemaker}
\end{figure}

\subsection{Amazon Bedrock}
Amazon Bedrock (logo riportato in Figura \ref{fig:bedrock}) è un servizio completamente gestito che offre una selezione di modelli di fondazione (\textit{foundation models}, FM) di alta qualità, provenienti da startup AI leader e da Amazon stessa, disponibili attraverso un'API unificata. Questo servizio consente di scegliere il modello più adatto alle specifiche esigenze di un caso d'uso e di creare applicazioni di intelligenza artificiale generativa con elevati standard di sicurezza, privacy e responsabilità. 

Con Amazon Bedrock, è possibile personalizzare privatamente i modelli di fondazione utilizzando tecniche come il fine-tuning e il \textit{Retrieval Augmented Generation} (RAG), integrandoli facilmente nelle applicazioni senza dover gestire infrastrutture. Tra i modelli disponibili vi è Claude di Anthropic, un modello avanzato per la generazione di testo. Amazon Bedrock supporta anche la creazione di agenti in grado di eseguire compiti utilizzando sistemi e fonti di dati aziendali, migliorando l'efficienza e la precisione delle applicazioni basate su AI generativa.

Nel contesto del presente progetto, Amazon Bedrock e in particolare il modello Claude non sono stati utilizzati direttamente, in quanto si è ritenuto l'utilizzo di Amazon Comprehend e Amazon Textract sufficiente per le esigenze di analisi del testo e dei documenti.

\begin{figure}[h]
  \centering
  \includegraphics[width=0.2\textwidth]{img/tecnologie/bedrock.png}
  \caption{Logo di Amazon Bedrock}
  \label{fig:bedrock}
\end{figure}

\section{Strumenti di sviluppo}
In aggiunta ai servizi AWS, sono stati utilizzati diversi strumenti di sviluppo per la realizzazione dell'applicazione. Questi strumenti hanno permesso di scrivere, testare e monitorare il codice. Di seguito sono elencati i principali strumenti utilizzati nel corso del progetto.
\subsection{Jupyter Notebook}
Jupyter Notebook (logo riportato in Figura \ref{fig:jupyter}) è un'applicazione web open-source che consente di creare e condividere documenti interattivi contenenti codice, testo, grafici e altri elementi multimediali. Jupyter Notebook supporta diversi linguaggi di programmazione, tra cui Python, R e Julia, e offre un ambiente di sviluppo flessibile e versatile per l'analisi dei dati, la visualizzazione e la prototipazione di modelli di machine learning.

Nel contesto del presente progetto, Jupyter Notebook è stato utilizzato per eseguire analisi preliminari sui dati, testare le funzionalità di Amazon Comprehend e Amazon Textract e sviluppare i modelli di classificazione.

\begin{figure}[h]
  \centering
  \includegraphics[width=0.3\textwidth]{img/tecnologie/jupyter.png}
  \caption{Logo di Jupyter Notebook}
  \label{fig:jupyter}
\end{figure}
\subsection{Visual Studio Code}
Visual Studio Code (logo riportato in Figura \ref{fig:vscode}) è un editor di codice sorgente sviluppato da Microsoft, disponibile per Windows, Linux e macOS. Grazie alla sua versatilità e alle numerose estensioni disponibili, Visual Studio Code è stato utilizzato per lo sviluppo del codice dell'applicazione, inclusi i \textit{Lambda functions} e i notebook. Inoltre, l'editor è stato impiegato per redigere e gestire la documentazione del progetto, sfruttando le sue funzionalità avanzate di editing e integrazione con strumenti di controllo di versione.


\begin{figure}[h]
  \centering
  \includegraphics[width=0.3\textwidth]{img/tecnologie/vscode.png}
  \caption{Logo di Visual Studio Code}
  \label{fig:vscode}
\end{figure}

\subsection{Git}
\glsfirstoccur{\gls{gitg}} è un sistema di controllo di versione distribuito ampiamente utilizzato per gestire e tracciare le modifiche al codice sorgente durante lo sviluppo software. Nel presente progetto, Git è stato utilizzato per monitorare l'evoluzione del codice sorgente.


\begin{figure}[h]
  \centering
  \includegraphics[width=0.3\textwidth]{img/tecnologie/git.png}
  \caption{Logo di Git}
  \label{fig:git}
\end{figure}

\subsection{Bitbucket}
Bitbucket è un servizio di hosting di \glsfirstoccur{\gls{repositoryg}} Git basato su cloud. Bitbucket è stato utilizzato per memorizzare il codice sorgente dell'applicazione.

\begin{figure}[h]
  \centering
  \includegraphics[width=0.5\textwidth]{img/tecnologie/bitbucket.png}
  \caption{Logo di Bitbucket}
  \label{fig:bitbucket}
\end{figure}

\section{Linguaggi di programmazione}
Nel corso del progetto sono stati utilizzati diversi linguaggi di programmazione per sviluppare le funzionalità dell'applicazione. Di seguito sono elencati i principali linguaggi utilizzati e le relative caratteristiche.
\subsection{Python}
Python è un linguaggio di programmazione ad alto livello, interpretato, adatto per lo sviluppo di applicazioni web, desktop e mobile. Python è stato utilizzato per la realizzazione delle funzioni Lambda.

\begin{figure}[h]
  \centering
  \includegraphics[width=0.2\textwidth]{img/tecnologie/python.png}
  \caption{Logo di Python}
  \label{fig:python}
\end{figure}