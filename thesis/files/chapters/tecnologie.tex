\chapter{Tecnologie}
\label{cap:tecnologie}
In questo capitolo verranno presentate le tecnologie utilizzate durante lo stage.
\section{Linguaggi di programmazione}
\subsection{Python}
Python è un linguaggio di programmazione ad alto livello, interpretato, adatto per lo sviluppo di applicazioni web, desktop e mobile. Python è stato utilizzato per la realizzazione del backend dell'applicazione.

\begin{figure}[h]
  \centering
  \includegraphics[width=0.2\textwidth]{img/tecnologie/python.png}
  \caption{Logo di Python}
  \label{fig:python}
\end{figure}



\section{Amazon Web Services}
Amazon Web Services (AWS) è una piattaforma di servizi cloud che offre potenza di calcolo, storage di database, distribuzione di contenuti e altre funzionalità per aiutare le aziende a scalare e crescere. AWS offre una vasta gamma di servizi che possono essere utilizzati per implementare soluzioni di Intelligenza Artificiale (AI) e Machine Learning (ML). Per la realizzazione dell'applicazione sono stati individuati diversi servizi che hanno permesso di realizzare un'archittettura scalabile e serverless. 

\subsection{Amazon Comprehend}
Amazon Comprehend è un servizio di analisi del linguaggio naturale (NLP) che utilizza l'apprendimento automatico per identificare informazioni utili nei testi. Amazon Comprehend è stato utilizzato per analizzare i testi delle recensioni degli utenti dell'applicazione.

\begin{figure}[h]
  \centering
  \includegraphics[width=0.2\textwidth]{img/tecnologie/comprehend.png}
  \caption{Logo di Amazon Comprehend}
  \label{fig:comprehend}
\end{figure}

\subsection{Amazon Textract}
Amazon Textract è un servizio di OCR (Optical Character Recognition) che utilizza l'apprendimento automatico per riconoscere e analizzare il testo e i dati delle immagini. Amazon Textract è stato utilizzato per estrarre il testo dalle immagini delle ricette.

\begin{figure}[h]
  \centering
  \includegraphics[width=0.2\textwidth]{img/tecnologie/textract.png}
  \caption{Logo di Amazon Textract}
  \label{fig:textract}
\end{figure}

\subsection{Amazon S3}
Amazon Simple Storage Service (Amazon S3) è un servizio di storage di oggetti che offre scalabilità, disponibilità dei dati, sicurezza e prestazioni. Amazon S3 è progettato per memorizzare grandi quantità di dati a un costo molto basso. Amazon S3 è stato utilizzato per memorizzare i file inerenti alle diversi fasi del progetto.

\begin{figure}[h]
  \centering
  \includegraphics[width=0.2\textwidth]{img/tecnologie/s3.png}
  \caption{Logo di Amazon S3}
  \label{fig:s3}
\end{figure}

\subsection{AWS Lambda}
AWS Lambda è un servizio di calcolo serverless che esegue il codice in risposta a eventi e gestisce automaticamente le risorse di calcolo richieste dal codice. AWS Lambda è stato utilizzato per implementare le funzioni di backend dell'applicazione. 

\begin{figure}[h]
  \centering
  \includegraphics[width=0.3\textwidth]{img/tecnologie/AWS_Lambda.png}
  \caption{Logo di AWS Lambda}
  \label{fig:lambda}
\end{figure}

\subsection{Amazon DynamoDB}
Amazon DynamoDB è un servizio di database NoSQL completamente gestito che offre prestazioni di singolo millisecondo a qualsiasi scala. Amazon DynamoDB è stato utilizzato per memorizzare i dati relativi ai vari utenti dell'applicazione.

\begin{figure}[h]
  \centering
  \includegraphics[width=0.3\textwidth]{img/tecnologie/DynamoDB.png}
  \caption{Logo di Amazon DynamoDB}
  \label{fig:dynamodb}
\end{figure}

\subsection{Amazon Step Functions}
AWS Step Functions è un servizio di orchestrazione di serverless che consente di coordinare facilmente i componenti di applicazioni distribuite e microservizi utilizzando logica visuale. 

\begin{figure}[h]
  \centering
  \includegraphics[width=0.3\textwidth]{img/tecnologie/stepfunctions.png}
  \caption{Logo di Amazon Step Functions}
  \label{fig:stepfunctions}
\end{figure}

\section{Strumenti di sviluppo}
\subsection{Visual Studio Code}
Visual Studio Code è un editor di codice sorgente sviluppato da Microsoft per Windows, Linux e macOS. Visual Studio Code è stato utilizzato per scrivere il codice dell'applicazione.

\begin{figure}[h]
  \centering
  \includegraphics[width=0.3\textwidth]{img/tecnologie/vscode.png}
  \caption{Logo di Visual Studio Code}
  \label{fig:vscode}
\end{figure}

\subsection{Git}
Git è un sistema di controllo di versione distribuito utilizzato per tenere traccia delle modifiche al codice sorgente durante lo sviluppo del software. Git è stato utilizzato per tenere traccia delle modifiche al codice sorgente dell'applicazione.

\begin{figure}[h]
  \centering
  \includegraphics[width=0.3\textwidth]{img/tecnologie/git.png}
  \caption{Logo di Git}
  \label{fig:git}
\end{figure}

\subsection{Bitbucket}
Bitbucket è un servizio di hosting di repository Git basato su cloud. Bitbucket è stato utilizzato per memorizzare il codice sorgente dell'applicazione.

\begin{figure}[h]
  \centering
  \includegraphics[width=0.5\textwidth]{img/tecnologie/bitbucket.png}
  \caption{Logo di Bitbucket}
  \label{fig:bitbucket}
\end{figure}