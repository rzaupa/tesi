\chapter{Conclusioni}
\label{cap:conclusioni}
\emph{In questa sezione vengono presentate le conclusioni del lavoro svolto.}

% per avere termine di glossario con apice "g" che appare
Lorem \glsfirstoccur{\gls{sdkg}}
\\
% per avere termine di glossario normale
Lorem \gls{apig}

\section{Consuntivo finale}

Le ore di stage effettivamente svolte sono state 320, rispettando il monte ore previsto. Il lavoro svolto è stato suddiviso in diverse fasi, ognuna delle quali ha richiesto un impegno specifico. La fase iniziale di studio e formazione ha richiesto un tempo maggiore rispetto a quanto preventivato, in quanto ho dovuto approfondire le tecnologie e gli strumenti necessari per lo sviluppo del progetto. La fase di progettazione e sviluppo ha richiesto un impegno costante, ma sono riuscito a rispettare i tempi previsti. Infine, la fase di test e validazione ha richiesto un tempo inferiore rispetto a quanto preventivato, in quanto il prodotto sviluppato ha funzionato correttamente fin da subito.

\section{Raggiungimento degli obiettivi}
Gli obiettivi prefissati all'inizio del percorso (riportati nella sezione \ref{cap:descrizione-stage}) di stage sono stati raggiunti con successo. 

%\begin{UC}
%    \usecase
%\end{UC}

\section{Conoscenze acquisite}
TO DO

\section{Valutazione personale}
In conclusione, ritengo di essere soddisfatto del lavoro svolto e delle competenze acquisite durante il percorso di stage. L'esperienza ha rappresentato un'opportunità di crescita professionale e personale, permettendomi di mettermi alla prova in un contesto lavorativo reale e di confrontarmi con problematiche complesse. Ho potuto apprendere concetti relativi al cloud computing e all'applicazione di servizi cloud, oltre che un particolare approfondimento sulle tecnologie di machine learning per l'elaborazione intelligente dei documenti. Inoltre, ho avuto modo di lavorare in un team di sviluppo, migliorando le mie capacità di collaborazione e di comunicazione. Infine, ho potuto mettere in pratica le competenze acquisite durante il percorso di studi, dimostrando di saper affrontare con successo le sfide che mi sono state proposte.
