% chktex-file 24

\chapter{Descrizione dello stage}
\label{cap:descrizione-stage}

%\intro{Breve introduzione al capitolo}\\

\section{Introduzione al progetto}

\begin{figure}[!ht]
    \centering
    \includegraphics[width=0.5\columnwidth]{pk_estate.jpg}
    \caption{Caption}
\end{figure}
\lipsum[1]

\section{Analisi preventiva dei rischi}

Durante la fase di analisi iniziale sono stati individuati alcuni possibili
rischi a cui si potrà andare incontro. Si è quindi proceduto a elaborare delle
possibili soluzioni per far fronte a tali rischi.

\begin{risk}{Performance del simulatore hardware}
    \riskdescription{le performance del simulatore hardware e la comunicazione con questo potrebbero risultare lenti o non abbastanza buoni da causare il fallimento dei test}
    \risksolution{coinvolgimento del responsabile a capo del progetto relativo il simulatore hardware}
    \label{risk:hardware-simulator}
\end{risk}

\section{Requisiti e obiettivi}
Gli obiettivi sono stati definiti in accordo con il tutor aziendale e si
identificano nel seguente modo:
\[
    \text{[Priorità][Id]}
\]
\begin{itemize}
    \item Priorità: indica la priorità dell'obiettivo, può essere obbligatorio o
          desiderabile;
    \item Id: identifica l'obiettivo in modo univoco rispetto alla priorità.
\end{itemize}

\begin{longtable}{|c|p{4cm}|p{10cm}|}
    \hline
    \textbf{ID}  & \textbf{Categoria}                                               & \textbf{Descrizione}                                       \\
    \hline
    \endfirsthead

    \hline
    \textbf{ID}  & \textbf{Categoria}                                               & \textbf{Descrizione}                                       \\
    \hline
    \endhead

    O01          & Obbligatorio                                                     & Analisi dei servizi AWS per l'addestramento dei modelli AI
    \\ \hline O02 & Obbligatorio & Addestramento di un modello di apprendimento AI
    utilizzando i servizi AWS                                                                                                                    \\ \hline O03 & Obbligatorio & Analisi requisiti
    applicativi e tecnici per implementare la soluzione richiesta                                                                                \\ \hline O04   &
    Obbligatorio & Implementare un modello di apprendimento automatico che analizzi
       il contenuto delle \glsfirstoccur{PEC} importate e assegni loro categorie appropriate in base
    al contenuto (mittente, destinatario, data e argomento)                                                                                      \\ \hline D01   &
    Desiderabile & Implementare algoritmi di IA in grado di adattarsi e apprendere
       continuamente dai dati per migliorare le prestazioni del sistema nel tempo. Ciò
       include l'ottimizzazione dei modelli di apprendimento automatico in base
    all'esperienza e ai feedback degli utenti                                                                                                    \\ \hline D02 & Desiderabile &
       Integrazione con un sistema documentale per l’archiviazione delle PEC creando i
       metadati necessari con le informazioni estratte e collocandole nella corretta
    categoria di appartenenza                                                                                                                    \\ \hline
\end{longtable}

\section{Pianificazione}
\subsection{Pianificazione settimanale}
Il periodo di stage è stato suddiviso in 8 settimane, durante le quali sono
previste le seguenti attività:
\begin{longtable}{|c|c|c|p{8cm}|}
    \hline
    \textbf{Settimana} & \textbf{Dal} & \textbf{Al} & \textbf{Attività}                                          \\
    \hline
    \endfirsthead

    \hline
    \textbf{Settimana} & \textbf{Dal} & \textbf{Al} & \textbf{Attività}                                          \\
    \hline
    \endhead

    1                  & 24-06-2024   & 28-06-2024  &
    - Incontro con persone coinvolte nel progetto per discutere i requisiti e le richieste di implementazione \newline
    - Ricerca, studio e documentazione per inquadramento progetto \newline
    - Introduzione ai linguaggi di sviluppo \newline
    - Introduzione agli ambienti di sviluppo \newline
    - Introduzione dei servizi \glsfirstoccur{awsg}                                                                        \\
    \hline
    2                  & 01-07-2024   & 05-07-2024  &
    - Analisi dei servizi AWS per l'addestramento di un modello di apprendimento \newline
    - Addestramento di un modello di apprendimento utilizzando i servizi di AWS \newline
    \textbf{Milestone:} Utilizzo dei servizi AWS per l'addestramento di un modello di apprendimento              \\
    \hline
    3                  & 08-07-2024   & 12-07-2024  &
    - Studio della soluzione per definire i requisiti necessari per l’implementazione \newline
    \textbf{Milestone:} Analisi dei requisiti applicativi e tecnici per implementare la soluzione                \\
    \hline
    4                  & 15-07-2024   & 19-07-2024  &
    - Addestramento modello di apprendimento per catalogare le PEC in base al loro contenuto                     \\
    \hline
    5                  & 22-07-2024   & 26-07-2024  &
    - Implementazioni per interfacciarsi con il modello di apprendimento addestrato e per poter catalogare le PEC importate \newline
    \textbf{Milestone:} Completamento obiettivi minimi                                                           \\
    \hline
    6                  & 29-07-2024   & 02-08-2024  &
    - Implementazione algoritmo di AI per l’autoapprendimento                                                    \\
    \hline
    7                  & 05-08-2024   & 09-08-2024  &
    - Studio e documentazione sulle API messe a disposizione dal documentale per poter catalogare le mail PEC \newline
    - Implementazione dell’integrazione con il documentale producendo i metadati necessari per catalogare le PEC \\
    \hline
    8                  & 12-08-2024   & 16-08-2024  &
    - Verifica e test archiviazione PEC nel documentale \newline
    \textbf{Milestone:} Completamento obiettivi massimi                                                          \\
    \hline
    9                  & 19-08-2024   & 23-08-2024  &
    - Recupero eventuali ritardi                                                                                 \\
    \hline
    10                 & 26-08-2024   & 30-08-2024  &
    - Recupero eventuali ritardi                                                                                 \\
    \hline

\end{longtable}