% chktex-file 24
\chapter{Descrizione dello stage}
\label{cap:descrizione-stage}

\emph{In questo capitolo verrà descritto il progetto di stage, analizzando il contesto aziendale e le attività svolte durante il periodo di stage.}

\section{Introduzione al progetto}

L'elaborazione intelligente dei documenti (\glsfirstoccur{\gls{idpg}}) è una tecnologia che automatizza il processo di immissione manuale dei dati da documenti cartacei o immagini digitali, integrandoli con altri processi aziendali digitali. Ad esempio, in un flusso di lavoro aziendale automatizzato, come l'invio di ordini ai fornitori al momento del calo delle scorte, l'\gls{idpg} può sostituire l'immissione manuale dei dati da parte del team contabile. Invece di inserire manualmente i dati di una fattura ricevuta via e-mail, i sistemi di \gls{idpg} estraggono automaticamente queste informazioni e le integrano direttamente nel sistema contabile, eliminando ostacoli e riducendo gli errori.

L'\gls{idpg} offre numerosi vantaggi alle aziende. In termini di \glsfirstoccur{\gls{scalabilitàg}}, permette di elaborare documenti su larga scala con precisione, evitando errori umani e aumentando l'efficienza operativa. Promuove una cultura dell'efficienza dei costi, automatizzando attività ripetitive e riducendo i costi associati all'elaborazione manuale dei dati. Migliora anche la soddisfazione dei clienti grazie alla gestione più rapida e automatizzata dei documenti, come l'onboarding, le prenotazioni e i pagamenti, consentendo di fornire risposte personalizzate e veloci ai clienti.

Diversi settori traggono beneficio dall'\gls{idpg}. Nel settore sanitario, facilita la gestione delle cartelle cliniche, migliorando l'estrazione e l'organizzazione dei dati dai documenti medici. Le aziende finanziarie lo utilizzano per automatizzare la gestione delle spese e l'elaborazione delle fatture, semplificando la gestione dei pagamenti. Nel settore legale, l'\gls{idpg} analizza contratti e documenti legali, utilizzando tecnologie di elaborazione del linguaggio naturale (\glsfirstoccur{\gls{nlpg}}) per estrarre informazioni chiave. Le aziende della logistica lo impiegano per tracciare spedizioni e documenti di transito, riducendo gli errori umani. Infine, nel settore delle risorse umane, l'\gls{idpg} semplifica la selezione del personale, gestisce le buste paga e automatizza altre funzioni HR.

Le tecnologie alla base dell'\gls{idpg} comprendono il riconoscimento ottico dei caratteri (\glsfirstoccur{\gls{ocrg}}), che converte immagini di testo in dati leggibili dalle macchine, e l'elaborazione del linguaggio naturale (\gls{nlpg}), che analizza e comprende il linguaggio umano. L'automazione robotica dei processi (RPA) consente invece di automatizzare i flussi di lavoro aziendali ripetendo azioni umane predefinite.

Il processo di IDP si articola in diverse fasi: acquisizione e classificazione dei documenti, estrazione dei dati rilevanti tramite \gls{ocrg} e \gls{nlpg}, convalida e successiva elaborazione dei dati nei sistemi aziendali, e apprendimento continuo attraverso algoritmi di machine learning per migliorare l'accuratezza nel tempo. Inoltre, i sistemi di \gls{idpg} offrono report e analisi per ottimizzare ulteriormente i flussi di lavoro aziendali.

\glsfirstoccur{\gls{awsg}} (AWS) supporta l'implementazione dell'\gls{idpg} attraverso servizi come Amazon Textract, che utilizza il machine learning per estrarre informazioni dai documenti senza interazioni manuali, e Amazon Comprehend, che sfrutta l'\gls{nlpg} per scoprire informazioni preziose nei testi. Entrambi i servizi consentono alle aziende di automatizzare la gestione dei documenti in modo efficiente e sicuro, integrandosi con altre piattaforme aziendali per un flusso di lavoro senza interruzioni.



%\begin{figure}[!ht]
%    \centering
%    \includegraphics[width=0.5\columnwidth]{pk_estate.jpg}
%    \caption{Caption}
%\end{figure}
%Digressione su idp.\\
%\lipsum[1]

%\section{Analisi preventiva dei rischi}
%
%Durante la fase di analisi iniziale sono stati individuati alcuni possibili
%rischi a cui si potrà andare incontro. Si è quindi proceduto a elaborare delle
%possibili soluzioni per far fronte a tali rischi.
%
%\begin{risk}{Performance del simulatore hardware}
%    \riskdescription{le performance del simulatore hardware e la comunicazione con questo potrebbero risultare lenti o non abbastanza buoni da causare il fallimento dei test}
%    \risksolution{coinvolgimento del responsabile a capo del progetto relativo il simulatore hardware}
%    \label{risk:hardware-simulator}
%\end{risk}
%
\section{Requisiti e obiettivi}
Gli obiettivi sono stati definiti in accordo con il tutor aziendale e si
identificano nel seguente modo:
\[
    \text{[Priorità][Id]}
\]
\begin{itemize}
    \item Priorità: indica la priorità dell'obiettivo, può essere obbligatorio o
          desiderabile;
    \item Id: composto da due cifre, identifica l'obiettivo in modo univoco rispetto alla priorità.
\end{itemize}

\begin{longtable}{|c|p{4cm}|p{10cm}|}
    \hline
    \textbf{ID}  & \textbf{Categoria}                                               & \textbf{Descrizione}                                       \\
    \hline
    \endfirsthead

    \hline
    \textbf{ID}  & \textbf{Categoria}                                               & \textbf{Descrizione}                                       \\
    \hline
    \endhead

    O01          & Obbligatorio                                                     & Analisi dei servizi \gls{awsg} per l'addestramento dei modelli \gls{aig}
    \\ \hline O02 & Obbligatorio & Addestramento di un modello di apprendimento \gls{aig}
    utilizzando i servizi \gls{awsg}                                                                                                                    \\ \hline O03 & Obbligatorio & Analisi requisiti
    applicativi e tecnici per implementare la soluzione richiesta                                                                                \\ \hline O04   &
    Obbligatorio & Implementare un modello di apprendimento automatico che analizzi
       il contenuto delle \gls{pecg} importate e assegni loro categorie appropriate in base
    al contenuto (mittente, destinatario, data e argomento)                                                                                      \\ \hline D01   &
    Desiderabile & Implementare algoritmi di \gls{aig} in grado di adattarsi e apprendere
       continuamente dai dati per migliorare le prestazioni del sistema nel tempo. Ciò
       include l'ottimizzazione dei modelli di apprendimento automatico in base
    all'esperienza e ai feedback degli utenti                                                                                                    \\ \hline D02 & Desiderabile &
       Integrazione con un sistema documentale per l’archiviazione delle \gls{pecg} creando i
       metadati necessari con le informazioni estratte e collocandole nella corretta
    categoria di appartenenza                                                                                                                    \\ \hline
\end{longtable}

\section{Pianificazione}
\subsection{Pianificazione settimanale}
Il periodo di stage è stato suddiviso in 8 settimane, durante le quali sono
previste le seguenti attività:
\begin{longtable}{|c|c|c|p{8cm}|}
    \hline
    \textbf{Settimana} & \textbf{Dal} & \textbf{Al} & \textbf{Attività}                                          \\
    \hline
    \endfirsthead

    \hline
    \textbf{Settimana} & \textbf{Dal} & \textbf{Al} & \textbf{Attività}                                          \\
    \hline
    \endhead

    1                  & 24-06-2024   & 28-06-2024  &
    - Incontro con persone coinvolte nel progetto per discutere i requisiti e le richieste di implementazione \newline
    - Ricerca, studio e documentazione per inquadramento progetto \newline
    - Introduzione ai linguaggi di sviluppo \newline
    - Introduzione agli ambienti di sviluppo \newline
    - Introduzione dei servizi \gls{awsg}                                                                       \\
    \hline
    2                  & 01-07-2024   & 05-07-2024  &
    - Analisi dei servizi \gls{awsg} per l'addestramento di un modello di apprendimento \newline
    - Addestramento di un modello di apprendimento utilizzando i servizi di \gls{awsg} \newline
    \textbf{Milestone:} Utilizzo dei servizi \gls{awsg} per l'addestramento di un modello di apprendimento              \\
    \hline
    3                  & 08-07-2024   & 12-07-2024  &
    - Studio della soluzione per definire i requisiti necessari per l’implementazione \newline
    \textbf{Milestone:} Analisi dei requisiti applicativi e tecnici per implementare la soluzione                \\
    \hline
    4                  & 15-07-2024   & 19-07-2024  &
    - Addestramento modello di apprendimento per catalogare le \gls{pecg} in base al loro contenuto                     \\
    \hline
    5                  & 22-07-2024   & 26-07-2024  &
    - Implementazioni per interfacciarsi con il modello di apprendimento addestrato e per poter catalogare le \gls{pecg} importate \newline
    \textbf{Milestone:} Completamento obiettivi minimi                                                           \\
    \hline
    6                  & 29-07-2024   & 02-08-2024  &
    - Implementazione algoritmo di \gls{aig} per l’autoapprendimento                                                    \\
    \hline
    7                  & 05-08-2024   & 09-08-2024  &
    - Studio e documentazione sulle \glsfirstoccur{\gls{apig}} messe a disposizione dal documentale per poter catalogare le mail \gls{pecg} \newline
    - Implementazione dell’integrazione con il documentale producendo i metadati necessari per catalogare le \gls{pecg} \\
    \hline
    8                  & 12-08-2024   & 16-08-2024  &
    - Verifica e test archiviazione \gls{pecg} nel documentale \newline
    \textbf{Milestone:} Completamento obiettivi massimi                                                          \\
    \hline
    9                  & 19-08-2024   & 23-08-2024  &
    - Recupero eventuali ritardi                                                                                 \\
    \hline
    10                 & 26-08-2024   & 30-08-2024  &
    - Recupero eventuali ritardi                                                                                 \\
    \hline

\end{longtable}