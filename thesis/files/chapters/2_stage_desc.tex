% chktex-file 24
\chapter{Descrizione dello \emph{stage}}
\label{cap:descrizione-stage}

\emph{In questo capitolo viene presentata una panoramica del progetto di stage, con una descrizione dettagliata del contesto aziendale, del progetto e degli obiettivi prefissati. Vengono inoltre elencati i requisiti e gli obiettivi del progetto, i prodotti attesi e la pianificazione delle attività.}

\section{Introduzione al progetto}

Le organizzazioni di diversi settori, come sanità, finanza, legale, retail e manifatturiero, gestiscono quotidianamente una grande quantità di documenti nei loro processi aziendali. Questi documenti contengono informazioni critiche, essenziali per prendere decisioni tempestive e mantenere alti livelli di soddisfazione del cliente, velocizzare l'\emph{onboarding} e ridurre il tasso di abbandono dei clienti. Nella maggior parte dei casi, l'elaborazione di tali documenti avviene manualmente, un processo che richiede tempo, è soggetto a errori, costoso e difficile da scalare. Inoltre, l'automazione attualmente disponibile per l'elaborazione dei documenti è limitata.

L'elaborazione intelligente dei documenti, detta anche \glsfirstoccur{\gls{idp}}, consente di automatizzare l'estrazione delle informazioni da documenti di diverso tipo e formato, in modo rapido e preciso, senza necessità di competenze avanzate di \gls{machinelearningg}. Questa tecnologia riduce i costi complessivi, migliorando al contempo l'efficienza e la qualità delle decisioni aziendali.

\gls{idpg} automatizza la raccolta e l'elaborazione delle informazioni dai documenti digitali o cartacei, integrandole nei flussi di lavoro aziendali digitali. Ad esempio, in un'azienda che invia ordini ai fornitori al calo delle scorte, \gls{idpg} sostituisce l'immissione manuale dei dati estraendo automaticamente le informazioni rilevanti dalle fatture ricevute via \emph{e-mail} e integrandole nel sistema contabile. Questo processo elimina gli ostacoli e riduce notevolmente gli errori, migliorando l'efficienza operativa.

\gls{idpg} offre numerosi vantaggi alle aziende, come la \glsfirstoccur{\gls{scalabilitàg}} nella gestione di grandi volumi di documenti, l'automazione delle attività ripetitive e la riduzione dei costi di elaborazione manuale. Inoltre, velocizza l'interazione con i clienti, automatizzando attività come l'\emph{onboarding} e la gestione dei pagamenti, garantendo risposte tempestive e personalizzate. 

I settori che beneficiano dell'\gls{idpg} includono la sanità, dove facilita la gestione delle cartelle cliniche e l'organizzazione dei dati medici; le finanze, dove automatizza la gestione delle spese e delle fatture; il settore legale, dove analizza contratti e documenti complessi; la logistica, dove migliora la tracciabilità delle spedizioni; e le risorse umane, dove semplifica la selezione del personale e gestisce le buste paga.

Le tecnologie principali che supportano l'\gls{idpg} includono il riconoscimento ottico dei caratteri (\glsfirstoccur{\gls{ocr}}), che converte le immagini di testo in dati leggibili dalle macchine, e l'elaborazione del linguaggio naturale (\glsfirstoccur{\gls{nlp}}), che analizza e comprende il linguaggio umano. L'automazione robotica dei processi (\glsfirstoccur{\gls{rpa}}) permette inoltre di ripetere azioni umane predefinite per automatizzare i flussi di lavoro aziendali.

Il processo di \gls{idpg} (vedi Figura~\ref{fig:idp_workflow}) segue tipicamente diverse fasi: acquisizione e classificazione dei documenti; estrazione dei dati tramite tecnologie come \gls{ocrg} e \gls{nlpg}; convalida e inserimento delle informazioni nei sistemi aziendali e apprendimento continuo attraverso modelli di \gls{machinelearningg} per migliorare l'accuratezza. I sistemi di \gls{idpg} forniscono anche \emph{report} e analisi che aiutano le aziende a ottimizzare ulteriormente i loro flussi di lavoro.

\begin{figure}[!ht]
    \centering
    \includegraphics[width=0.8\columnwidth]{idp_workflow.png}
    \caption{Flusso di lavoro dell'elaborazione intelligente dei documenti}
    \label{fig:idp_workflow}
\end{figure}

In un contesto aziendale sempre più orientato all'innovazione, l'elaborazione dei documenti ha subito notevoli trasformazioni grazie all'introduzione dell'\gls{idpg}, che trasforma i dati non strutturati presenti in vari tipi di documenti in informazioni strutturate e fruibili, migliorando drasticamente l'efficienza e riducendo lo sforzo manuale. Tuttavia, il potenziale dell'\gls{idpg} non si esaurisce qui. Con l'integrazione dell'intelligenza artificiale generativa (\glsfirstoccur{\gls{generativeaig}}), l'\gls{idpg} può essere ulteriormente potenziata.

L'integrazione di modelli di linguaggio di grandi dimensioni (\glsfirstoccur{\gls{llmg}}) nelle architetture \gls{idpg} consente di ottenere capacità avanzate di estrazione dati, adattandosi dinamicamente ai cambiamenti nei modelli dei dati. \gls{aws} supporta questa evoluzione con strumenti come \emph{Amazon Textract}, un servizio di \gls{machinelearningg} che estrae automaticamente testo, scrittura e dati da documenti scansionati, e \emph{Amazon Bedrock}, un servizio completamente gestito che offre modelli di \gls{aig} di base attraverso \glsfirstoccur{\gls{apig}} di facile utilizzo. Un ulteriore strumento chiave è \emph{Amazon Comprehend}, un servizio di \gls{nlpg} che consente di analizzare e comprendere il contenuto dei documenti, estraendo informazioni chiave, come entità, sentiment e concetti, direttamente dal testo.

In particolare, l'integrazione di \emph{Amazon Textract} con LangChain, utilizzato come document loader, e \emph{Amazon Bedrock}, per l'estrazione di dati e capacità di \gls{aig}, permette di estendere le funzionalità di una nuova o esistente architettura \gls{idpg}. \emph{Amazon Comprehend} si combina perfettamente in questo flusso, utilizzando \gls{nlpg} per analizzare il contenuto dei documenti e fornire \emph{insight} dettagliati, aumentando la precisione dell'estrazione delle informazioni e permettendo di ottenere analisi più approfondite.

Questa combinazione introduce non solo un'automazione più avanzata nell'elaborazione dei documenti, ma anche una capacità di adattamento e miglioramento continuo grazie all'\gls{aig}, consentendo di affrontare in modo dinamico i modelli di dati in evoluzione.

In conclusione, l'integrazione di \gls{idpg}, \gls{aig} e strumenti offerti da \gls{awsg} come \emph{Amazon Textract}, \emph{Amazon Comprehend} e \emph{Amazon Bedrock} rappresenta una nuova frontiera nell'elaborazione documentale, offrendo alle aziende non solo efficienza operativa, ma anche flessibilità e capacità di risposta alle nuove sfide del mercato.

Nel contesto del progetto di \emph{stage}, l'obiettivo è implementare un modello di \gls{aig} per l'elaborazione intelligente degli allegati delle \gls{pecg} e, per perseguire a questo scopo, si è deciso di utilizzare i servizi \gls{awsg} per il flusso di lavoro dell'elaborazione intelligente dei documenti.

%\begin{figure}[!ht]
%    \centering
%    \includegraphics[width=0.5\columnwidth]{pk_estate.jpg}
%    \caption{Caption}
%\end{figure}
%Digressione su idp.\\
%\lipsum[1]

%\section{Analisi preventiva dei rischi}
%
%Durante la fase di analisi iniziale sono stati individuati alcuni possibili
%rischi a cui si potrà andare incontro. Si è quindi proceduto a elaborare delle
%possibili soluzioni per far fronte a tali rischi.
%
%\begin{risk}{Performance del simulatore hardware}
%    \riskdescription{le performance del simulatore hardware e la comunicazione con questo potrebbero risultare lenti o non abbastanza buoni da causare il fallimento dei test}
%    \risksolution{coinvolgimento del responsabile a capo del progetto relativo il simulatore hardware}
%    \label{risk:hardware-simulator}
%\end{risk}
%
\section{Requisiti e obiettivi}

Gli obiettivi del progetto sono stati definiti in accordo con il tutor aziendale e si identificano univocamente nel modo seguente:
\[
    \text{[Priorità][Id]}
\]
\begin{itemize}
    \item \textbf{Priorità}: indica il livello di importanza dell'obiettivo, che può essere \emph{Obbligatorio} o \emph{Desiderabile};
    \item \textbf{Id}: composto da due cifre, specifica in modo univoco l'obiettivo rispetto alla priorità.
\end{itemize}

Di seguito è riportata una tabella che elenca i requisiti e gli obiettivi stabiliti per il progetto.

\begin{longtable}{|c|p{4cm}|p{10cm}|}
    \hline
    \textbf{ID}  & \textbf{Categoria}                                               & \textbf{Descrizione}                                       \\
    \hline
    \endfirsthead

    \hline
    \textbf{ID}  & \textbf{Categoria}                                               & \textbf{Descrizione}                                       \\
    \hline
    \endhead

    O01          & Obbligatorio                                                     & Analisi dei servizi \gls{awsg} per l'addestramento dei modelli di \gls{aig}. \\
    \hline 
    O02          & Obbligatorio                                                     & Addestramento di un modello di apprendimento \gls{aig} utilizzando i servizi \gls{awsg}. \\
    \hline 
    O03          & Obbligatorio                                                     & Analisi dei requisiti applicativi e tecnici per implementare la soluzione richiesta. \\
    \hline 
    O04          & Obbligatorio                                                     & Implementazione di un modello di \gls{machinelearningg} che analizzi il contenuto delle \gls{pecg} importate e assegni loro categorie appropriate (mittente, destinatario, data e argomento). \\
    \hline 
    D01          & Desiderabile                                                     & Implementazione di algoritmi di \gls{aig} in grado di adattarsi e apprendere continuamente dai dati per migliorare le prestazioni del sistema nel tempo. Questo include l'ottimizzazione dei modelli di apprendimento in base all'esperienza e ai \emph{feedback} degli utenti. \\
    \hline 
    D02          & Desiderabile                                                     & Integrazione con un sistema documentale per l'archiviazione delle \gls{pecg}, creando i metadati necessari e assegnando le informazioni estratte alla corretta categoria. \\
    \hline
    \caption{Tabella dei requisiti e obiettivi dello stage} \\
\end{longtable}

\subsection{Prodotti attesi}

Tra i principali risultati attesi dal progetto, vi è la produzione di una relazione scritta che illustri in dettaglio i punti chiave del lavoro svolto. In particolare, la relazione include:

\begin{itemize}
    \item Una contestualizzazione del progetto, che spieghi il problema affrontato e gli obiettivi perseguiti;
    \item Un'analisi completa e approfondita che copra:
    \begin{itemize}
        \item L'inquadramento generale del progetto;
        \item I requisiti applicativi e tecnici;
        \item La struttura del \emph{database};
        \item Gli strumenti e le applicazioni di terze parti utilizzati;
        \item I principali casi d'uso individuati.
    \end{itemize}
    \item Uno studio di fattibilità, volto a dimostrare la possibilità di implementare la soluzione proposta;
    \item La descrizione dettagliata dell'implementazione della soluzione sviluppata.
\end{itemize}

\subsection{Contenuti formativi previsti}

Nel corso di questo progetto di \emph{stage}, si avrà l'opportunità di approfondire diverse aree tecniche e migliorare le proprie competenze. In particolare, gli ambiti di conoscenza che saranno oggetto di apprendimento includono:

\begin{itemize}
    \item \textbf{Linguaggi e strumenti tecnologici}: familiarità con una serie di tecnologie chiave, tra cui:
    \begin{itemize}
        \item L'uso del \emph{database MySQL}, se necessario per il progetto;
        \item I servizi \emph{cloud} di \gls{awsg}, utilizzati per l'addestramento dei modelli di \gls{ai} e il framework \gls{awsg} \glsfirstoccur{\gls{sdkg}};
        \item \emph{Python} e gli strumenti di analisi forniti da \emph{notebook Jupyter}.
    \end{itemize}
    \item \textbf{Competenze trasversali}: esperienza pratica nel lavoro di gruppo, in particolare all'interno di un \emph{team} che utilizza il framework \gls{agileg} \gls{scrumg}, migliorando così le proprie capacità di collaborazione e gestione del lavoro in un contesto aziendale strutturato.
\end{itemize}

\section{Pianificazione}

La pianificazione delle attività è stata organizzata in base agli obiettivi prefissati e alle scadenze stabilite. La tabella seguente riporta la pianificazione settimanale delle attività svolte durante il periodo di \emph{stage}.

\begin{longtable}{|c|c|c|p{8cm}|}
    \hline
    \textbf{Settimana} & \textbf{Dal} & \textbf{Al} & \textbf{Attività}                                          \\
    \hline
    \endfirsthead

    \hline
    \textbf{Settimana} & \textbf{Dal} & \textbf{Al} & \textbf{Attività}                                          \\
    \hline
    \endhead

    1                  & 24-06-2024   & 28-06-2024  &
    Incontro con le persone coinvolte nel progetto per discutere i requisiti e le richieste di implementazione; ricerca, studio e documentazione per l'inquadramento del progetto; introduzione ai linguaggi di sviluppo, agli ambienti di sviluppo e dei servizi \gls{awsg}. \\
    \hline
    2                  & 01-07-2024   & 05-07-2024  &
    Analisi dei servizi \gls{awsg} per l'addestramento di un modello di \gls{machinelearningg}. \newline \textbf{Milestone:} Utilizzo dei servizi \gls{awsg} per l'addestramento di un modello di apprendimento. \\
    \hline
    3                  & 08-07-2024   & 12-07-2024  &
    Studio della soluzione per definire i requisiti necessari per l’implementazione. \newline \textbf{Milestone:} Analisi dei requisiti applicativi e tecnici per implementare la soluzione. \\
    \hline
    4                  & 15-07-2024   & 19-07-2024  &
    Addestramento del modello di apprendimento per catalogare le \gls{pecg} in base al loro contenuto. \\
    \hline
    5                  & 22-07-2024   & 26-07-2024  &
    Implementazione per interfacciarsi con il modello di apprendimento addestrato e per catalogare le \gls{pecg} importate.\newline \textbf{Milestone:} Completamento degli obiettivi minimi. \\
    \hline
    6                  & 29-07-2024   & 02-08-2024  &
    Implementazione dell'algoritmo di \gls{aig} per l’autoapprendimento. \\
    \hline
    7                  & 05-08-2024   & 09-08-2024  &
    Studio e documentazione sulle \gls{apig} messe a disposizione dal sistema documentale per catalogare le \gls{pecg}; implementazione dell’integrazione con il documentale producendo i metadati necessari per catalogare le \gls{pecg}. \\
    \hline
    8                  & 12-08-2024   & 16-08-2024  &
    Verifica e test dell'archiviazione delle \gls{pecg} nel documentale.\newline \textbf{Milestone:} Completamento degli obiettivi massimi. \\
    \hline
    9                  & 19-08-2024   & 23-08-2024  &
    Recupero di eventuali ritardi. \\
    \hline
    10                 & 26-08-2024   & 30-08-2024  &
    Recupero di eventuali ritardi. \\
    \hline
    \caption{Tabella della pianificazione dello stage} \\
\end{longtable}

Durante lo svolgimento dello \emph{stage}, sono state applicate modifiche alla pianificazione in base alle necessità e alle richieste del tutor aziendale.

\section{Organizzazione del lavoro}

Lo stage si è svolto nel periodo dal 24 giugno 2024 al 30 agosto 2024, con una durata complessiva di 8 settimane e un totale di 320 ore di lavoro. È stata prevista una modalità mista, con 2 giorni a settimana di presenza in azienda e 3 giorni in modalità remota, con impegno full-time ogni giorno. La sede di riferimento è stata il Centro per la Formazione di \myCompany, situato in Via dell'Edilizia, 100, Vicenza (VI).

L'inserimento è avvenuto in un gruppo di lavoro, con il supporto continuo del \emph{team} e del tutor aziendale. Il tutor è stato spesso presente nel gruppo, garantendo una modalità di interazione costante e facilitando il processo di revisione e \emph{feedback}. Le revisioni del progetto sono state condotte in conformità alla metodologia \gls{scrumg}, con brevi riunioni giornaliere di 5 minuti e una revisione settimanale della durata di 1 ora.

L'azienda ha fornito strumenti di comunicazione e collaborazione come \emph{Google Meet} per le riunioni e \emph{Google Drive} per la condivisione dei documenti, organizzati attraverso un gruppo creato su \emph{Gmail}. La comunicazione in modalità \emph{smart working} è avvenuta tramite la \emph{chat} di \emph{Google Chat}, garantendo un canale diretto per il dialogo tra lo studente e il tutor aziendale.

Oltre ai punti sopra elencati, durante il corso dello \emph{stage} sono stati svolti due incontri con il tutor aziendale, denominati SAL (Stato Avanzamento Lavoro), per discutere lo stato di avanzamento del progetto e valutare eventuali modifiche da apportare. Il primo incontro è stato svolto a metà \emph{stage}, mentre il secondo è avvenuto alla conclusione dello \emph{stage}. L'incontro di chiusura è stato utilizzato per discutere i risultati ottenuti e per valutare il lavoro complessivo svolto.