% chktex-file 2
% chktex-file 8
% chktex-file 11
% chktex-file 24
% chktex-file 26
\chapter{Introduzione}
\label{cap:introduzione}

%\begin{figure}[h!]
%    \centering
%    \includegraphics[width=1\columnwidth]{img/quantum_entanglement.png}
%    \caption{Lorem}
%    \label{fig:entanglement}
%\end{figure}
%
%Introduzione al contesto applicativo.
%
%Lorem Figure \ref{fig:entanglement}
%
%Esempio di utilizzo di un termine nel glossario \gls{api}.
%
%Esempio di citazione in linea
%\cite{site:agile-manifesto}.
%
%Esempio di citazione nel pie' di pagina
%citazione\footcite{womak:lean-thinking}
%
%Termine di glossario \gls{apig}
%
%\lipsum[1-2]


\section{L'azienda}

\myCompany è un'azienda italiana di sviluppo software e consulenza informatica. Da quarant'anni supporta la riorganizzazione dei processi in tutte le aree aziendali, progettando e realizzando soluzioni digitali integrate. 
\newline
L'azienda che ad oggi conta più di 600 dipendenti e oltre 2500 aziende seguite ha come sede principale Villa Ramanelli a Grisignano di Zocco, in provincia di Vicenza, poco distante dai Centri di Ricerca e Sviluppo (CRS) e dal Centro per la Formazione di Vicenza. Conta anche diverse filiati in Trentino-Alto Adige, Friuli-Venezia Giulia, Lombardia, Piemonte, Emilia-Romagna, Toscana, Campania e Puglia. 
\newline
L'azienda è organizata in Business Unit, dei centri di competenza specifici e autonomi ma in relazione costante. Ognuna delle quali è specializzata in un settore specifico. 
\newline
L'obiettivo principale è l'innovazione e il progresso tecnologico, con l'obiettivo di creare soluzioni software che siano in grado di rispondere alle esigenze dei clienti, garantendo la massima qualità e sicurezza.
\newline
La metodologia di lavoro, indipendentemente dalla Business Unit, è basata su un approccio Agile implementata con il framework Scrum. Agile è un approccio alla gestione dei progetti che si basa su principi di collaborazione, auto-organizzazione e flessibilità. Scrum è un framework Agile che permette di gestire progetti complessi, garantendo la massima trasparenza e la massima flessibilità.
\newline
Ulteriori informazioni sono disponibili sul sito web dell'azienda\footnote{\url{https://www.sanmarcoinformatica.com/}}.
\begin{figure}[h!]
    \centering
    \includegraphics[width=0.5\columnwidth]{img/logo_sanmarco_informatica.png}
    \caption{Logo di Sanmarco Informatica}
    \label{fig:entanglement}
\end{figure}
\section{L'offerta di stage}
L'idea dello stage consiste nella catalogazione delle Poste Elettroniche Certificate (PEC) integrando tecnologie di Intelligenza Artificiale (AI) per l'analisi e l'efficienza del processo. 
\newline
Il modello di apprendimento automatico analizza il contenuto delle PEC e le classifica in base al contenuto. 
\newline
Il progetto è stato proposto dall'azienda in occasione dell'evento Stage IT 2024, organizzato dall'Università degli Studi di Padova e promosso da Confindustria Veneto Est. Tale evento mira ad agevolare l'incontro tra studenti e aziende, offrendo la possibilità di svolgere uno stage formativo con specifico riferimento al settore ICT.

\section{Organizzazione del testo}

\begin{description}
    \item[{\hyperref[cap:processi-metodologie]{Il secondo capitolo}}] descrive ...
    
    \item[{\hyperref[cap:descrizione-stage]{Il terzo capitolo}}] approfondisce ...
    
    \item[{\hyperref[cap:analisi-requisiti]{Il quarto capitolo}}] approfondisce ...
    
    \item[{\hyperref[cap:progettazione-codifica]{Il quinto capitolo}}] approfondisce ...
    
    \item[{\hyperref[cap:verifica-validazione]{Il sesto capitolo}}] approfondisce ...
    
    \item[{\hyperref[cap:conclusioni]{Nel settimo capitolo}}] descrive ...
\end{description}

Riguardo la stesura del testo, relativamente al documento sono state adottate le seguenti convenzioni tipografiche:
\begin{itemize}
	\item gli acronimi, le abbreviazioni e i termini ambigui o di uso non comune menzionati vengono definiti nel glossario, situato alla fine del presente documento;
	%\item per la prima occorrenza dei termini riportati nel glossario viene utilizzata la seguente nomenclatura: \emph{parola}\glsfirstoccur;
	\item i termini in lingua straniera o facenti parti del gergo tecnico sono evidenziati con il carattere \emph{corsivo}.
\end{itemize}
